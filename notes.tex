\documentclass[12pt]{article} % change to 'article' for standard fonts
\usepackage{notes}
\usepackage{libertine}

%$Additional packages
\usepackage{bbm}
\usepackage{mathrsfs}
\usepackage{nicefrac}
\renewcommand{\emptyset}{\varnothing}

%%Cal letters
\newcommand{\CG}{\mathcal{G}}
\newcommand{\CU}{\mathcal{U}}
\newcommand{\CX}{\mathcal{X}}
\newcommand{\CY}{\mathcal{Y}}
\newcommand{\CC}{\mathcal{C}}
\newcommand{\CD}{\mathcal{D}}
\newcommand{\CH}{\mathcal{H}}
\newcommand{\CI}{\mathcal{I}}
\newcommand{\CL}{\mathcal{L}}
\newcommand{\CM}{\mathcal{M}}
\newcommand{\CP}{\mathcal{P}}
\newcommand{\CR}{\mathcal{R}}
\newcommand{\CS}{\mathcal{S}}
\newcommand{\CZ}{\mathcal{Z}}
\newcommand{\CT}{\mathcal{T}}
\newcommand{\CB}{\mathcal{B}}
\newcommand{\CA}{\mathcal{A}}
\newcommand{\CW}{\mathcal{W}}
\newcommand{\CF}{\mathcal{F}}
\newcommand{\CO}{\mathcal{O}}
\newcommand{\CQ}{\mathcal{Q}}
\newcommand{\CN}{\mathcal{N}}
\newcommand{\CE}{\mathcal{E}}

%%Scr letters
\newcommand{\SG}{\mathscr{G}}
\newcommand{\SU}{\mathscr{U}}
\newcommand{\SX}{\mathscr{X}}
\newcommand{\SY}{\mathscr{Y}}
\newcommand{\SC}{\mathscr{C}}
\newcommand{\SD}{\mathscr{D}}
\newcommand{\SH}{\mathscr{H}}
\newcommand{\SI}{\mathscr{I}}
\newcommand{\SL}{\mathscr{L}}
\newcommand{\SM}{\mathscr{M}}
\newcommand{\SP}{\mathscr{P}}
\newcommand{\SR}{\mathscr{R}}
\renewcommand{\SS}{\mathscr{S}}
\newcommand{\SZ}{\mathscr{Z}}
\newcommand{\ST}{\mathscr{T}}
\newcommand{\SB}{\mathscr{B}}
\newcommand{\SA}{\mathscr{A}}
\newcommand{\SW}{\mathscr{W}}
\newcommand{\SF}{\mathscr{F}}
\newcommand{\SO}{\mathscr{O}}
\newcommand{\SQ}{\mathscr{Q}}
\newcommand{\SN}{\mathscr{N}}

%%Category thoery
\newcommand{\obj}{\on{obj}}
\newcommand{\Set}{\on{Set}}
\newcommand{\Top}{\on{Top}}
\newcommand{\hTop}{\on{hTop}}
\newcommand{\Grp}{\on{Grp}}
\newcommand{\Ab}{\on{Ab}}
\newcommand{\Rng}{\on{Rng}}
\newcommand{\Hom}{\on{Hom}}
\newcommand{\id}{\on{id}}

%%Var
\newcommand{\vp}{\varphi}
\newcommand{\vr}{\varrho}

%%Mathbb letters
\newcommand{\N}{\mathbb{N}}
\newcommand{\R}{\mathbb{R}}
\newcommand{\Q}{\mathbb{Q}}
\newcommand{\C}{\mathbb{C}}
\newcommand{\Z}{\mathbb{Z}}
\newcommand{\F}{\mathbb{F}}

%%Operators
\newcommand{\on}[1]{\operatorname{#1}}
\newcommand{\ons}[1]{\operatorname{\mathsf{#1}}}
\DeclareMathOperator*{\argmin}{arg min}
\DeclareMathOperator*{\argmax}{arg max}
\newcommand{\intset}{\on{int}}
\newcommand{\extset}{\on{ext}}

%%Useful characters
\newcommand{\eps}{\varepsilon}

%%Vectors, random variables
\renewcommand{\b}[1]{\boldsymbol{\mathbf{#1}}}
\newcommand{\proj}{\on{Proj}}

%%Probability
\newcommand{\E}{\mathbf{E}}
\renewcommand{\Pr}{\mathbf{P}}
\newcommand{\ind}{\mathbbm{I}}

%%Linear algebra
\newcommand{\rank}{\on{rank}}
\newcommand{\spanset}{\on{span}}
\newcommand{\conv}{\on{conv}}

%%Analysis
\newcommand{\Rloc}{\CR_{\on{loc}}}
\newcommand{\BV}{\on{BV}}
\newcommand{\NBV}{\on{NBV}}
 
% infinitesimal
\renewcommand{\d}{\, \mathrm{d}}
% upper/lower integrals
\def\upint{\mathchoice
    {\mkern13mu\overline{\vphantom{\intop}\mkern7mu}\mkern-20mu}
    {\mkern7mu\overline{\vphantom{\intop}\mkern7mu}\mkern-14mu}
    {\mkern7mu\overline{\vphantom{\intop}\mkern7mu}\mkern-14mu}
    {\mkern7mu\overline{\vphantom{\intop}\mkern7mu}\mkern-14mu}
  \int}
\def\lowint{\mkern3mu\underline{\vphantom{\intop}\mkern7mu}\mkern-10mu\int}

%%Maps
\newcommand{\surjto}{\twoheadrightarrow}
\newcommand{\injto}{\hookrightarrow}
% Map restriction
\newcommand\restr[2]{{% we make the whole thing an ordinary symbol
  \left.\kern-\nulldelimiterspace % automatically resize the bar with \right
  #1 % the function
  \vphantom{\big|} % pretend it's a little taller at normal size
  \right|_{#2} % this is the delimiter
  }}

%%Learning theory
\newcommand{\err}{\on{err}}

%%Notetaking
\newcommand{\vocab}[1]{\textbf{\color{blue}\sffamily #1}}
\providecommand{\alert}{\vocab}
\newcommand{\hrulebar}{
  \par\hspace{\fill}\rule{0.95\linewidth}{.7pt}\hspace{\fill}
  \par\nointerlineskip \vspace{\baselineskip}
}

%%Useful notation
\newcommand{\ol}{\overline}
\newcommand{\ul}{\underline}
\newcommand{\wt}{\widetilde}
\newcommand{\wh}{\widehat}
\def\independenT#1#2{\mathrel{\rlap{$#1#2$}\mkern2mu{#1#2}}}
\newcommand{\dperp}{\protect\mathpalette{\protect\independenT}{\perp}}
\newcommand{\unif}[1]{\textsf{Unif}(#1)}
\newcommand{\sgn}{\on{sgn}}

\newcommand{\norm}[1]{\left\| #1\right\|}


\renewcommand{\b}[1]{\ensuremath{\mathbf{#1}}} % bold letters
\newcommand{\bs}[1]{\ensuremath{\boldsymbol{#1}}} % bold symbols
\newcommand{\bi}[1]{\ensuremath{\bm{#1}}} % bold symbols
\renewcommand{\c}[1]{\ensuremath{\mathcal{#1}}} % caligraphic font
\newcommand{\Ex}[1]{\ensuremath{\mathbb{E}[#1]}}  % expectation operator

\usepackage{geometry}
\geometry{left=.75in, right=.75in, top=1in, bottom=1in}

%%Optional packages
\usepackage[round,sort]{natbib}
\usepackage[algo2e,ruled,linesnumbered]{algorithm2e}
\usepackage{setspace}
\usepackage{subcaption}
\usepackage{booktabs}
\usepackage{chngpage}

%%Editing
\usepackage{color-edits}
\addauthor[Dutch]{dh}{cyan}
\addauthor[edit]{e}{red}

%%Document setup
\title{525a}
\author{\textsc{Dutch Hansen}}
\date{\normalsize{Fall 2024}}

\begin{document}

\maketitle

\pagebreak

\tableofcontents

\pagebreak

\section{Preliminaries}

Lectures by Prof. Juhi Jang. We use \cite{folland1999real}. Basic knowledge of set theory, real numbers, and extended reals assumed. Collaboration acceptable, but solo work encouraged.

\subsection{Motivation}

\begin{example}[Failures of the Riemann integral]
    Consider the sequence of Riemann-integrable fucntions $(f_n)$ such that $f_n \to f$, and suppose there exists some Riemann-intergrable $g$ such that $|f_n| \leq g$ everywhere. If $(f_n)$ is difficult to understand, maybe we can approximate with $g$. Is it true that $f$ is Riemann-integrable? Not in general: $f$ may be extremely irregular. We have the following example. 

    Order the rationals $\{r_k\} = \Q \cap [0,1]$, and define \[h_k(x) := \begin{cases}
        1, &x = r_k\\
        0, &\text{else}.
    \end{cases}\] Let $f_n(x) = \sum_{k=1}^{n} h_k(x)$. For each $n$, the number of discontinuities is finite, so $f_n$ is Riemann-integrable, but we claim its pointwise limit is not. Indeed, $f_n \to f$ for \[f(x) := \begin{cases}
        1, &x \in \Q\\
        0, &\text{else}.
    \end{cases}\] The limits of the upper and lower sums will always differ. As we will see, $f$ is actually Lebesgue-integrable w.r.t. the Lebesgue measure. 
\end{example}

\begin{example}[Generalized functions]
    Measure theory will also yield some generalization of functions. Consider the sequence $(f_n)$ for indicator function \[f_n(x) = n \cdot \chi_{[0, 1/n]}(x).\] This is certainly Riemann-integrable for every $n$. What about the limit? Take any $\phi \in C_c^{\infty}(\R)$, a smooth function with compact support. One can show that as $n \to \infty$, we have \[\int_\R f_n(x) \phi(x) \d x \to \phi(0).\] Interestingly, however, there is no function $f$ such that for arbitrary $\phi \in C_c^{\infty}(\R)$, \[\phi(0) = \int_\R f(x) \phi(x) \d x.\] To better understand this situation, we need ``measures,'' or generalized functions. In particular, we will study the Dirac measure $\delta_0(x)$.
\end{example}

\begin{example}[Convergence properties and completeness]
    Recall that a bounded sequence $(x_n)$ in $\R^d$ has a convergent subsequence by completeness. We'd like to have a similar statement for integrable functions. To do so, the notion of convergence must be changed. Often times, pointwise convergence will fail to give this property, but a different notion of convergence using the Lebesgue measure will give this property.
\end{example}

\subsection{Sets}

We follow the prologue of \cite{folland1999real}. Consider a faimly of sets $\{E_n\}_{n=1}^{\infty}$ indexed by $\N$. We define \[\limsup E_n = \cap_{k=1}^{\infty} \cup_{n=k}^{\infty} E_n, \ \ \ \ \ \ \liminf E_n = \cup_{k=1}^{\infty} \cap_{n=k}^{\infty} E_n.\] One can verify that \begin{align*}
    \limsup E_n &= \{x : x \in E_n \ \text{for infinitely many $n$} \}\\
    \liminf E_n &= \{x : x \in E_n \ \text{for all but finitely many $n$} \}
\end{align*}

Let $f : X \to Y$. For any indexed collection of subsets $\{E_\alpha\}$, it holds that \[f^{-1}\left(\bigcup_{\alpha} E_\alpha \right) = \bigcup_{\alpha} f^{-1} (E_\alpha), \ \ \ \ f^{-1}\left(\bigcap_{\alpha} E_\alpha \right) = \bigcap_{\alpha} f^{-1} (E_\alpha), \ \ \ \ f^{-1}(E^c) = \left(f^{-1}(E)\right)^c.\] That is, the inverse image commutes with union, intersection, and complement.

\subsection{Orderings}

\begin{definition}[Partial order]
    A partial ordering on a nonempty set $X$ is a relation on $X$ satisfying reflexivity, transitivity, and antisymmetry. A total (or linear) ordering on $X$ is a partial ordering on $X$ where the additional property holds: for all $x, y \in X$, either $x \sim y$ or $y \sim x$. We will typically denote partial orderings by $\leq$.
\end{definition}

\begin{definition}[Order isomorphic]
    Two posets $(X, \leq)$, $(Y, \leq)$ with order are called order isomorphic of there exists a bijection $X \to Y$ such that $x_1 \leq x_2$ iff $f(x_1) \leq f(x_2)$.
\end{definition}

\begin{definition}
    For poset $(X, \leq)$, a maximal (resp. minimal) element $x \in X$ is such that the only $y \in X$ satisfying $x \leq y$ (resp. $x \geq y$) is $x$ itself. Such elements may not exist, and they may not be unique unless the ordering is total.

    If $E \subset X$, an upper (resp. lower) bound for $E$ is an element $x \in X$ such that $y \leq x$ (resp. $x \leq y$) for all $y \in E$. Such a bound need not be an element of $E$.
\end{definition}

\begin{definition}[Well ordered]
    If $X$ is totally ordered by $\leq$, and every nonempty subset $E \subset X$ has a (necessarily unique) minimal element, $X$ is said to be well ordered by $\leq$. In this case, we say $\leq$ is called a well ordering on $X$. 
\end{definition}

\begin{principle}[Hausdorff Maximal Principle]
    Every partially ordered set has a maximal totally ordered subset. This is, if $X$ is partially ordered by $\leq$, there is a set $E \subset X$ that is totally ordered by $\leq$, such that no subset of $X$ that properly includes $E$ is totally ordered by $\leq$.
\end{principle}

\noindent An equivalent statement is Zorn's lemma.

\begin{principle}[Zorn's Lemma]
    If $X$ is a partially ordered set and every totally ordered subset of $X$ has an upper bound, then $X$ has a maximal element.
\end{principle}

\begin{proof}[Proof that Hausdorff's Principle is equivalent to Zorn's Lemma]
    Assume Hausdorff's, and let $X$ be a poset such that every totally ordered subset has an upper bound. Choose the maximal totally ordered subset $C$, and take its upper bound $x \in C$ (in $C$ by totaly ordering). We claim that $x$ is maximal in $X$; indeed, if not, then $C \cup \{y\}$ also forms a totally ordered subset for some $y \notin C$, contradicting maximality of $C$.

    To see the other direction, take the collection $Q$ of all totally ordered subset of $X$, which is partially ordered by set inclusion. Take $C$ to be a totally ordered subset of $Q$, a collection of subsets, each totally ordered by $\leq$, which are together totally ordered by $\subset$. Define $U = \bigcup_{S \in C} S$, the union of sets in $C$. It's clear that $U$ is a superset of any set in $C$. It is also easy to see that $U$ is itself totally orderd by $\leq$. For any $x, y \in U$, $x \in S_1$ and $y \in S_2$; since $Q$ it totally ordered, however, w.l.o.g. $S_1 \subset S_2$, so $x, y \in S_2$ and $x, y$ are comparable. Hence, $U$ is an upper bound for $C$ in $Q$. Since $C$ was arbitrary, we have satisfied the condition for Zorn's Lemma, and $Q$ has a maximal element, giving Hausdorff's Maximal Principle.
\end{proof}

\begin{principle}[Well Ordering Principle]
    Every nonempty set $X$ can be well ordered.
\end{principle}

\begin{proof}
    Let $\CW$ be the collection of well orderings of subsets of $X$, and define a partial ordering $\leq_\CW$ on $\CW$ as follows. If $\leq_1$ and $\leq_2$ are well orderings on the subsets $E_1$ and $E_2$, then $\leq_1$ precedes $\leq_2$ in the partial ordering if (i) $\leq_2$ extends $\leq_1$, meanining $E_1 \subset E_2$ and $\leq_2$ agrees with $\leq_1$ on $E_1$, and (ii) if $x \in E_2 \setminus E_1$ then $y_{\leq_2} x$ for all $y \in E_1$. One can check that this is a partial ordering.
    
    We aim to apply Zorn's lemma to $\CW$. Let $C$ be a totally ordered subset of $\CW$; we must produce some well-ordering $u = (U, \leq_U) \in \CW$ such that $x \leq_\CW u$ for all $x \in C$. Start by letting $C$ be a chain and defining \[S = \{E \subset X : E \ \ \text{is well-ordered by element of C}\}.\] Now take the union of all these subsets, \[U = \bigcup_{E \in S} E.\] Now define $\leq_U$ by $x \leq_U y$ whenever $x, y \in K$ and $x \leq_K y$ for some $(K, \leq_K) \in C$. 
    
    To see that $u \in \CW$, take an element and nonempty subset $a \in A \subset U$. By definition of $U$, we have $a \in (W, \leq_W) \in C$ for some $W$, and there is a minimal element $m \in W \cap S$ since $W$ is well-ordered. Now for arbitrary $a' \in A$, if $a' \in W$, it follows that $m \leq_W a'$, and if $a' \notin W$, then $a' \in (W', \leq_{W'}) \in C$. But since $C$ is a chain we have $(W, \leq_W) \leq_{\CW} (W', \leq_{W'})$, and so $m \leq_{W'} a'$. Since we take $\leq_U$ as the union of orders, $m \leq_U a'$, and $m$ is minimal in $A$. That is, $u \in \CW$. \dhcomment{Come back and check that $u$ is actually totally ordered, which is necessary for well-ordering.}

    To see that $u$ is an upper bound for $C$ in $\CW$, note that for any $(W, \leq_W) \in C$, $W \subset U$. Next, note that by defintion of $\leq_U$, $\leq_U$ agrees with $\leq_W$. Finally, for any $w' \in U \setminus W$ and $w \in W$, we have $w' \in W'$ for some $W'$ where $W' \setminus W \neq \emptyset$. Now by total ordering of $C$, it must be so that $W \leq_\CW W'$, so $w \leq_U w'$. Hence, $C$ is an upper bound for $C$. 

    The hypothesis of Zorn's Lemma is satisfied, so $\CW$ has a maximal element, and this must be a well ordering on $X$. Indeed, if $\leq$ is a well ordering on a proper subset $E$ of $X$ and $x_0 \in X \setminus E$, we can extend $\leq$ to a well ordering on $E \cup \{x_0\}$ by declaring that $x \leq x_0$ for all $x \in E$. 
\end{proof}

\begin{principle}[Axiom of Choice]
    If $\{X_\alpha\}_{\alpha \in A}$ is a nonempty collection of nonempty sets, then $\prod_{\alpha \in A} X_\alpha$ is nonempty.
\end{principle}

\begin{proof}[Proof by well-ordering principle]
    Let $X = \bigcup_{\alpha \in A} X_\alpha$. Pick a well ordering on $A$, and for $\alpha \in A$, let $f(\alpha)$ be the minimal element of $X_\alpha$. Then $f \in \prod_{\alpha \in A} X_\alpha$.
\end{proof}

\begin{corollary}
    If $\{X_\alpha\}_{\alpha \in A}$ is a disjoint collection of nonempty sets, there is a set $Y \subset \bigcup_{\alpha \in A} X_\alpha$ such that $Y \cap X_\alpha$ contains precisely one element for each $\alpha \in A$.
\end{corollary}

\begin{proof}
    Take $Y = f(A)$ where $f \in \prod_{\alpha \in A} X_\alpha$.
\end{proof}

\noindent While we have deduced the Axiom of Choice from the Hausdorff principle, the two are in fact equivalent.

\subsection{The Extended Reals}

\subsubsection{Standard Topology}

Recall that the standard topology on $\R$ is the topology whose basis is all open intervals. Considering $\ol{\R}$, let \[\CB_1 = \{(a, b) :  a, b \in \R, a < b\}, \ \ \ \CB_2 = \{[-\infty, a) : a \in \R\}, \ \ \ \CB_3 = \{(b, + \infty] : b \in \R\}.\] Now let $\ol{\CB} = \CB_1 \cup \CB_2 \cup \CB_3$. Since this collection covers $X$ and satisfies the intersection property, it is a basis for a topology $\ol{\CT}$, which we refer to as the standard topology on $\ol{\R}$. This topology is also compactible with the standard topology $\CT$ on $\R$. 

\begin{proposition}
    Let $\CT$ and $\ol{\CT}$ be the standard topologies for $\R$ and $\ol{\R}$, respectively. Let $E \subset \ol{\R}$. Then $E \in \ol{\CT}$ if and only if $E \cap \R \in \CT$.
\end{proposition}

\noindent It should be noted that this topology is metrizable, but the metric is very different from that which generates the standard topology on $\R$.

\subsubsection{Infinite Limits and Limits at Infinity}

Most of our work amounts to studying familiar definitions with the special case where $\pm \infty$ is involved. Many of the claims have mechanical proofs, which are omitted.

\begin{proposition}
    Let $\ol{\R}$ have the standard topology. Then \begin{enumerate}
        \item If $U$ is a neighborhood of $+\infty$ in $\ol{\R}$, theen there exists $M \in \R$ such that $(M, +\infty] \subset U$.
        \item If $A \subset \R$ and $A$ is not bounded above in $\R$, then $+\infty$ is a limit point of $A$ with respect to $\ol{\R}$.
    \end{enumerate}
\end{proposition}

\subsubsection{asdf}

From completeness, it follows that every sequence $(a_n)_{n=1}^{\infty}$ in $\ol{\R}$ has a limit inferior and limit superior. \[\limsup(a_n) := \inf_{k \geq 1} \left(\sup_{n \geq k} a_n \right), \ \ \ \ \ \ \liminf(a_n) := \sup_{k \geq 1} \left( \inf_{n \geq k} a_n \right).\]
A sequence $(a_n)_{n=1}^{\infty}$ converges in $\R$ if and only if these two sequences are equal and finite, in which case the limit is their common value. One can also define these notions for functions $f : \R \to \ol{\R}$, for instance, \[\limsup_{x \to a} f(x) = \inf_{\delta > 0} \left( \sup_{0 < |z-a| < \delta} f(x) \right).\]

Uncountable sums are occasionally relevant. For an arbitrary set $X$ and function $f : X \to [0,\infty]$, we define \[\sum_{x \in X} f(x) = \sup \left\{\sum_{x \in F} f(x) : F \subset X, \ \text{$F$ finite}\right\}.\]

\begin{proposition}
    Given $f : X \to [0,\infty]$, let $A = \{x : f(x) > 0\}$. If $A$ is uncountable, then $\sum_{x \in X}f(x) = \infty$. If $A$ is countably infinite, then $\sum_{x \in X} f(x) = \sum_{n=1}^{\infty}f(g(n))$ where $g : \N \to A$ is any bijection and the sum on the right is an ordinary infinite series.
\end{proposition}

\begin{proof}
    We have $A = \bigcup_{1}^{\infty} A_n$ where $A_n = \{x : f(x) > 1/n\}$. If $A$ is uncountable, there must be some uncountable $A_n$, and $\sum_{x \in F}f(x) > \on{card}(F) / n$ for any finite subset $F \subset A_n$. It follows that $\sum_{x \in X}f(x) = \infty$.

    If $A$ is countably infinite, $g : \N \to A$ is a bijection, and for $B_N := g(\{1, \ldots, N\})$, every finite subset $F$ of $A$ is contained in some $B_N$. Hence, \[\sum_{x \in F} f(x) \leq \sum_{n=1}^{N} f(g(n)) \leq \sum_{x \in X} f(x).\] Taking the supremum over $N$, we find \[\sum_{x \in F} f(x) \leq \sum_{n=1}^{\infty} f(g(n)) \leq \sum_{x \in X} f(x),\] and then taking the supremum over $F$, we obtain the desired result.
\end{proof}

If a function $f : \R \to \R$ is increasing, then $f$ has right- and left-hand limits at each point: \[f(a+) = \lim_{x \searrow a} f(x) = \inf_{x > a} f(x), \ \ \ \ \ \ f(a-) = \lim_{x \nearrow a} f(x) = \sup_{x < a} f(x).\] Moreover, the limiting values $f(\infty) = \sup_\R f(x)$ and $f(-\infty) = \inf_\R f(x)$ exist (and are possibly equal to $\pm \infty$). We say $f$ is right continuous if $f(a) = f(a+)$ for all $a \in \R$ and left continuous if $f(a) = f(a-)$ for all $a \in \R$.

\section{Measures}

\subsection{Introduction}

\subsubsection{Existence of non-measurable sets}

We want to measure arbitrary sets. We naively try to do this by constructing a map \[\mu : 2^{\R^n} \to [0, \infty] \subset \ol{\R}.\] We also want this map to have some nice properties: \begin{enumerate}
    \item For a countable collection of sets $E_1, E_2, \ldots$ which are disjoint, \[\mu\left( \bigcup_{i}E_i \right) = \sum_i \mu(E_i).\]
    \item If sets $E$ and $F$ are congruent (can be mapped to each other by translations, rotations, reflections), then $\mu(E) = \mu(F)$.
    \item $\mu(Q) = 1$ where $Q$ is the half-open\footnote{Choosing half-open has the benefit of easy partitions. Since we can decompose $\wt{Q} = [0,2)^n$ into $2^n$ sets with the same measure as $Q$, so that $\mu(\wt{Q}) = 2^n$.} unit cube, $Q = [0, 1)^n$.
\end{enumerate}

\noindent We may want to construct a $\mu$ by approximating arbitrary sets $U$ with several sets like $Q$. We will see that there is no such $\mu$ satisfying these desiderata. This is due to the existence of non-measurable sets.

\begin{lemma}
    There exists a set $N \subset [0,1]$ such that for any $\mu$ satisfying (i - iii), $\mu(N)$ is ill-defined.
\end{lemma}

\begin{proof}
    Consider the following equivalence relation on $\R$: \[x \sim y \iff x - y \in \Q.\] One can see that $\R$ decomposes into uncountably many $[x]$ in the partition. Indeed, $x$ can only be related to countably many $y$. By the Axiom of Choice, we can find some $N \subset [0,1)$ such that $N$ contains exactly one representative of every equivalence class.

    Now let $\{r_j\}$ be an enumeration of the rational numbers $\Q \cap (-1,1)$ and consider the set \[M = \bigcup_{j \in \N} \Big\{x + r_j \in \R : x \in N \Big\} = \bigcup_{i \in \N} N_{r_j}.\] Delaying a proof, we claim that (1) $[0,1) \subset M \subset [-1,2)$, (2) $N_{r_i} \cap N_{r_j} = \emptyset$ if $r_i \neq r_j$. Assuming this claim, we have \begin{align}\label{eq:non-measurable-set-contradiction}
        \mu(M) = \sum_{j \in \N} \mu(N_{r_j}) = \sum_{j \in \N} \mu(N),
    \end{align} and by monotonicity, \[\mu([0,1)) = 1 \leq \mu(M) \leq \mu([-1,2)) \underset{\text{(i)}}{=} \mu([-1,0)) + \mu([0,1)) + \mu([1,2)) \underset{\text{(ii, iii)}}{=} 3.\] But this is if course a contradiction because $\mu(N) > 0$, so \cref{eq:non-measurable-set-contradiction} implies $\mu(M) = \infty$.

    We now return to proving (1) and (2). For (1), first note that $N \subset [0,1)$, so $M = \bigcup_j N + r_j \subset [-1,2)$ since each $r_j \in (-1,1)$. Next, let $x \in [0,1)$. Then by the definition of $N$, there exists some $\wt{x} \in N$ such that $x \sim \wt{x}$. We have $x - \wt{x} \in \Q \cap (-1, 1)$, and $x-\wt{x} = r_{j_0}$ for some $j_0$ and this implies $x \in N + r_{j_0} = N_{r_j} \subset M$. For (2), assume there exists $x, y \in N$ such that \[x + r_i = y + r_j.\] Then $x - y \in \Q$, so $x \sim y$, and because $N$ contains exactly one representative, we have $x = y$, and $r_i = r_j$ necessarily.
\end{proof}

Our motivating question becomes: how can we rectify this issue? Are we asking too much from $\mu$? Or are we asking for too large of a domain? To better understand this situation, we cite but do not prove an illustrative result.

\begin{theorem}[Banach-Tarski, 1924]
    Suppse $U, V$ are two bounded open sets in $\R^n$, $n \geq 3$. Then there exists $k 
    \in \N$ and sets $E_1, \ldots, E_k \subset \R^n$, $F_1, \ldots, F_k \subset \R^n$ such that \begin{itemize}
        \item the $E_i$ are pairwise disjoint, and the $F_i$ are pairwise disjoint;
        \item $\bigcup E_i = U$ and $\bigcup F_i = V$;
        \item for each $j$, $E_j$ is congruent to $F_j$.
    \end{itemize}
\end{theorem}

\noindent In particular, we can cut a small set $U$ into finitely many pieces and rearrange them to build a very large open set $V$. The consequence for us is that we cannot have some $\mu : 2^{\R^n} \to [0,\infty]$ which assigns positive values to bounded open sets and satisfies (i) for finite sequences of sets. We will ultimately restrict the domain in our definition of a measure.

\subsection{$\sigma$-Algebras}

We attempt to resolve the issues found above through the $\sigma$-algebra.

\begin{definition}[Algebra]
    Let $X$ be a nonempty set. Then a non-empty collection of subsets $\SA \subset 2^X$ is called an algebra if \begin{enumerate}
        \item for a finite collection $E_1, \ldots, E_n \in \SA$, we have $\bigcup E_i \in \SA$;
        \item if $E \in \SA$, then the complement $E^c \in \SA$.
    \end{enumerate}
\end{definition}

\begin{definition}[$\sigma$-Algebra]
    An algebra closed under countable unions is called a $\sigma$-algebra.
\end{definition}

\begin{remark}
    Some obvious statements are: \begin{itemize}
        \item any algebra contains $\emptyset$ and $X$ itself;
        \item any algebra (resp. $\sigma$-algebra) is closed under finite (resp. countable) inetersections;
        \item given the presence of condition (ii) in the defintion of algebra, we can actually relax the additional condition of $\sigma$-algebra to hold for only disjoint unions. This can be seen by letting \begin{align*}
            F_k := E_k \setminus \left( \bigcup_{i < k} E_i \right) = E_k \cap \left( \bigcup_{i < k} E_i \right)^c = \left( E_k^c \bigcup \left( \bigcup_{i < k} E_i \right) \right)^c.
        \end{align*} We can construct all $F_k$ like so to be disjoint, and then $\bigsqcup F_k = \bigcup E_k$.
    \end{itemize}
\end{remark}

\begin{lemma}
    Let $\SA$ be an algebra over $X$. Then the following statements are equivalent. \begin{enumerate}
        \item For any countable sequence of sets $E_1, E_2, \ldots \in \SA$, we have $\bigcup E_i \in \SA$.
        \item For any countable sequence of \textit{disjoint} sets, $\bigsqcup E_i \in \SA$.
        \item For any countable \textit{increasing} sequence of sets (meaning $E_i \subset E_{i+1}$), we have $\bigcup E_i \in \SA$.
    \end{enumerate}
\end{lemma}

\begin{proof}
    Homework 1.
\end{proof}

\begin{example}
    \begin{enumerate}
        \item Let $X$ be any set. Then $2^X$ and $\{\emptyset, X\}$ are $\sigma$-algebras.
        \item If $\{\SA_\alpha\}_{\alpha \in A}$ is a family of $\sigma$-algebras, then $\bigcap_{\alpha} \SA_\alpha$ is also a $\sigma$-algbera.
    \end{enumerate}
\end{example}

\begin{proposition}
    If $\CE \subset 2^X$, there exists a unique, minimal $\sigma$-algebra $\SM(\CE)$ containing $\CE$, which is obtained by taking intersection of all $\sigma$-algebras containing $\CE$. $\SM(\CE)$ is the $\sigma$-algebra generated by $\CE$.
\end{proposition}

% \begin{remark}
%     ``The word trivial depends on who you are. If anything is not clear, try to justify yourself.'' - JJ
% \end{remark}

\begin{lemma}
    If $\CE \subset \SM(\CF)$, then $\SM(\CE) \subset \SM(\CF)$.
\end{lemma}

\begin{proof}
    Note that $\SM(\CF)$ is a $\sigma$-algebra containing $\CE$, but $\SM(\CE)$ is the intersection of all $\sigma$-algebra containing $\CE$.
\end{proof}

\dhcomment{can be constructed with tansfinite induction}

\subsubsection{Metric Spaces}

\begin{definition}[Borel $\sigma$-algebra]
    Let $(X, \vr)$ be a metric space. Then the $\sigma$-algebra generated by all open sets in $X$ is called the Borel $\sigma$-algebra on $X$. We denote this algebra $\SB_X$, and call its members Borel sets. This is defined, more generally, for any topological space $(X, \zeta)$.
\end{definition}

\begin{remark}
    By definition of the $\sigma$-algebra, the Borel $\sigma$-algebra will contain all open sets and closed sets, thus countable intersections of open sets, and countable unions of closed sets.
\end{remark}

\begin{definition}
    For concinnity, we define the following.
    \begin{itemize}
        \item A $G_\delta$ set is a countable intersection of open sets.
        \item An $F_\sigma$ set is a countable union of closed sets.
        \item A $G_{\delta, \sigma}$ set is a countable union of $G_\delta$ sets.
        \item An $F_{\sigma, \delta}$ set is a countable intersection of $F_\sigma$ sets.
    \end{itemize}
\end{definition}

\begin{proposition}
    The Borel $\sigma$-algebra on $\R$ is generated by each of the following:\begin{enumerate}
        \item the open intervals $\CE_1 = \{(a, b) : a < b\}$,
        \item the closed intervals $\CE_2 = \{[a, b] : a < b\}$,
        \item the half-open intervals $\CE_3 = \{(a, b] : a < b\}$ or $\CE_4 = \{[a, b) : a < b\}$,
        \item the open rays $\CE_5 = \{(a, \infty) : a \in \R \}$ or $\CE_6 = \{(-\infty, a) : a \in \R \}$,
        \item the closed rays $\CE_7 = \{[a, \infty) : a \in \R\}$ or $\CE_8 = \{(-\infty, a] : a \in \R\}$.
    \end{enumerate}
\end{proposition}

\begin{proof}
    We prove (i). We first show that (1) $\CE_i \subset \SB_\R$, so $\SM(\CE_i) \subset \SB_\R$. Then we show that (2) all open sets are contained in $\SM(\CE_i)$. To see (1) for $\CE_1$, note that all sets in $\CE_1$ are open. To see (2), note that all open sets in $\R$ can be written as a countable union of disjoint open intervals.

    We prove (ii). All elements of $\CE_2$ are closed and their complements are open. Hence, $\CE_2 \subset \SB_\R$, and $\SM(\CE_2) \subset \SB_\R$. On the other hand, any open interval $(a, b)$ can be written \[(a, b) = \bigcup_{n \in \N} \left[a + \frac{1}{n}, b - \frac{1}{n}\right].\] Hence, $\CE_1 \subset \CE_2$, so by (i), $\CB_\R = \SM(\CE_1) \subset \SM(\CE_2)$.

    We prove (iii). Note that $\CE_3, \CE_4$ consist of $G_\delta$ sets \[(a, b] = \bigcap_{n \in \N}\left(a, b + \frac{1}{n}\right).\] On the other hand, \[(a, b) = \bigcup_{n \in \N} \left(a, b - \frac{1}{n}\right].\]
    The rest is left as an exercise.
\end{proof}

\begin{remark}
    We used the fact that any open subset of $\R$ can be written as a countable disjoint union of open intervals. 
\end{remark}

\begin{proof}
    Let $U$ be open in $\R$ and $x \in U$. Let $I_x$ be the largest open interval such that $x \in I_x \subset U$. That is, take $I_x = (a_x, b_x)$ for $a_x = \inf\{a < x : (a, x) \subset U\}$ and $b_x = \sup \{ b > x : (x, b) \subset U\}$. It is clear that \[U = \bigcup_{x \in U} I_x,\] but we also claim this can be written as a dijoint union. Suppose $I_x \cap I_y \neq \emptyset$. Then $I_x \cup X_y \subset U$ is an open interval in $U$. Since $I_x$ is maximal, $I_x = I_x \cap I_y = I_y$. That is, any two distinct intervals $I_x$ must be disjoint. Choosing $\CI$ to be the collection of distinct $I_x$, we have \[U = \bigsqcup_{I_x \in \CI} I_x.\] Moreover, since the intervals are not singletons, they must contain rationals. But since $\Q$ is countable, $\CI$ is countable.
\end{proof}

\subsubsection{Product $\sigma$-Algebras}

We will want to understand $\R^n$, not just $\R$. This motivates a definition for the product algebra.

\begin{definition}
    Let $\{X_\alpha\}_{\alpha \in A}$ be an indexed collection of nonempty sets. We define the product of sets $X = \prod_{\alpha \in A} X_\alpha$ as the set of mappings $\psi : A \to \bigcup_{\alpha \in A} X_\alpha$ such that $\psi(\alpha) \in X_\alpha$ for all $\alpha$. We also define $\pi_\alpha : X \to X_\alpha$ to be the projection map, sending $\psi$ to $\psi(\alpha)$.

    If $\SM_\alpha$ is a $\sigma$-algebra on $X_\alpha$ for each $\alpha \in A$, then the product $\sigma$-algebra on $\prod_{\alpha \in A} X_\alpha$ is the $\sigma$-algebra generated by the set \[\left\{ \pi_{\alpha}^{-1}(E_\alpha) : E_\alpha \in \SM_\alpha, \alpha \in A\right\}.\] We denote this $\sigma$-algebra by $\bigotimes_{\alpha \in A} \SM_\alpha$.
\end{definition}

\begin{proposition}
    If $A$ is countable, then $\bigotimes_{\alpha \in A} \SM_\alpha$ is the $\sigma$-algebra generated by \[G = \left\{ \prod_{\alpha \in A} E_\alpha : E_\alpha \in \SM_\alpha\right\}.\]
\end{proposition}

\begin{proof}
    We want to show \[\bigotimes_{\alpha \in A} \SM_\alpha = \SM(G).\]
    If $E_\alpha \in \SM_\alpha$, then $\pi_{\alpha}^{-1}(E_\alpha) = \prod_{\beta \in A} E_\beta$ such that $E_\beta = X_\beta$ if $\beta \neq \alpha$. This is contained in $\SM(G)$. On the other hand, note that \[\prod_{\alpha \in A}E_\alpha = \bigcap_{\alpha \in A} \pi_{\alpha}^{-1}(E_\alpha).\] Since $\bigotimes_{\alpha \in A} \SM_\alpha$ is closed under countable intersection, we have $\prod_{\alpha \in A}E_\alpha \in \bigotimes_{\alpha \in A} \SM_\alpha$. It follows that $G \subset \bigotimes_{\alpha \in A} \SM_\alpha$ and hence $\SM(G) \subset \bigotimes_{\alpha \in A} \SM_\alpha$.
\end{proof}

\begin{proposition}\label{prop:countable-sigma-algebra-generator}
    Suppose $\SM_\alpha$ is generated by $\CE_\alpha, \alpha \in A$. Then $\bigotimes_{\alpha \in A} \SM_\alpha$ is generated by \[\CF_1 = \left\{ \pi_{\alpha}^{-1}(E_\alpha) : E_\alpha \in \CE_\alpha, \alpha \in A \right\}.\] And if $A$ is countable and $X_\alpha \in \CE_\alpha$ for all $\alpha \in A$, then $\bigotimes_{\alpha \in A} \SM_\alpha$ is generated by \[\CF_2 = \left\{ \prod_{\alpha \in A} E_\alpha : E_\alpha \in \CE_\alpha \right\}.\]
\end{proposition}

\begin{proof}
    To see the first point, note that \[\CF_1 \subset \left\{ \pi_{\alpha}^{-1}(E_\alpha) : E_\alpha \in \SM_\alpha, \alpha \in A\right\},\] so $\SM(\CF_1) \subset \bigotimes_{\alpha \in A} \SM_\alpha$.

    Next, for each $\alpha \in A$, one can check that \[Y = \left\{ E \subset X_\alpha : \pi_{\alpha}^{-1}(E) \in \SM(\CF_1) \right\}\] is a $\sigma$-algebra on $X_\alpha$ (easy check), and it contains $\CE_\alpha$. Therefore $Y \supset \SM_\alpha$. Thus, $\pi_{\alpha}^{-1}(E) \in \SM(\CF_1)$ for all $E \in \SM_\alpha, \alpha \in A$. Hence, $\bigotimes_{\alpha \in A} \SM_\alpha \subset \SM(\CF_1)$. 
    
    The second claim follows from the first, as it now suffices to show $\SM(\CF_1) = \SM(\CF_2)$. If we let $\pi_{\alpha}^{-1}(E_\alpha) \in \CF_1$, then $\pi_{\alpha}^{-1}(E_\alpha) = \prod_{\beta}E_\beta$ where $E_\beta = X_\beta$ for all $\beta \neq \alpha$, and $X_\beta \in \CE_\beta$ by assumption. In particular, $\CF_1 \subset \CF_2$, so $\SM(\CF_1) \subset \SM(\CF_2)$. Next, let $\prod E_\alpha \in \CF_2$, and note that $\prod E_\alpha = \bigcap \pi_{\alpha}^{-1}(E_\alpha)$. Since $\SM(\CF_1)$ is closed under countable intersections, and $A$ is be countable, $\prod E_\alpha \in \SM(\CF_1)$.
\end{proof}

\begin{lemma}
    Let $(X, \vr)$ be a separable metric space. Any open set $O \subset X$ can be written as a countable union of open balls with rational radii.
\end{lemma}

\begin{proposition}
    Let $X_1, \ldots, X_n$ be metric spaces and $X = \prod_{j=1}^{n} X_j$ equipped with the product metric. Then $\bigotimes_{j} \SB_{X_j} \subset \SB_X$. Moreover, if each $X_j$ is separable, then $\bigotimes_{j} \SB_{X_j} = \SB_X$.
\end{proposition}

\begin{proof}
    See \href{https://math.stackexchange.com/questions/1943686/in-separable-metric-space-every-open-set-is-at-most-countable-union-of-open-ball}{Stack Exchange}.
\end{proof}

\begin{proof}
    By the most recent proposition, $\bigotimes \SB_{X_j}$ is generated by the sets $\pi_{j}^{-1}(U_j)$ where $U_j$ is an open set in $X_j$. Since these sets are open in the product topology (which is induced by the product metric), $\bigotimes \SB_{X_j} \subset \SB_X$.

    Next, suppose that for each $j$, $C_j$ is a countable dense set in $X_j$. Moreover, let $\CE_j$ denote the set of balls in $X_j$ with rational radius and center in $C_j$. Every open set in $X_j$ is a union of members of $\CE_j$---in fact a countable union since $\CE_j$ itself is countable. It follows that $\SB_{X_j} = \SM(\CE_j)$. Since $[n]$ is countable, we have that \[\bigotimes \SB_{X_j} = \SM\left(\left\{\prod_{j=1}^{n} E_j : E_j \in \CE_j \cup X_j\right\}\right) = \SM\left(\left\{\prod_{j=1}^{n} E_j : E_j \in \CE_j\right\}\right).\] In the remainder of the proof, we show that $\SB_X$ is generated by the same set.
    
    Note that the set of points in $X$ whose $j$th coordinate is in $C_j$ for all $j$ is a countable dense subset of $X$, and the balls of radius $r$ in the $X$ are products of balls of radius $r$ in the $X_j$'s. The set of open balls can then be generated by countable unions in $\{\prod_{j=1}^{n} E_j : E_j \in \CE_j \}$, so $\SB_X$ is generated by this set.
\end{proof}

\begin{corollary}
    $\SB_{\R^n} = \bigotimes_{i=1}^{n} \SB_\R$.
\end{corollary}

\begin{definition}
    A collection $\CE \subset 2^X$ is an elementary family if \begin{enumerate}
        \item $\emptyset \in \CE$,
        \item if $E, F \in \CE$, then $E \cap F \in \CE$,
        \item if $E \in \CE$, then $E^c$ is a finite disjoint union of members of $\CE$.
    \end{enumerate}
\end{definition}

\begin{proposition}
    If $\CE$ is an elementary family, the collection $\CA$ of finite disjoint unions of members of $\CE$ is an algebra.
\end{proposition}

\begin{proof}
    Let $A, B \in \CE$. Since $B^c = \sqcup_{j=1}^{J} C_j$ for $C_j$ in $\CE$, then \[A \cup B = (A \setminus B) \sqcup B = \left(\bigcup A \cap C_j \right) \cup B,\] which is a finite disjoint union of members in $\CE$. So $A \cup B \in \CA$. This argument may be extended to show that any finite union of sets in $\CE$ is in $\CA$. It follows that $\CA$ is closed under finite union.

    Now let $A = \cup_{i=1}^{n} A_i \in \CA$ for disjoint $\{A_i\}_{i=1}^{n} \subset \CE$. We have that $A_m^c = \sqcup_{j=1}^{J_m} B_{m}^{j}$, and the collections $\{B_{m}^{j}\}_{j=1}^{J_m}$ have disjoint members of $\CE$. Then \[ A^c = \left( \bigcup_{m=1}^{n} A_m \right)^c = \bigcap_{m=1}^{n} \left( \bigcup_{j=1}^{J_m} B_{m}^{j} \right) = \bigcup \Big\{ B_1^{j_1} \cap \cdots \cap B_{n}^{j_n} : j_m \in [J_m], \ \forall m \in [n] \Big\}.\] That is, $A^c \in \CA$.
\end{proof}

\subsection{Measures}

\begin{definition}
    Let $X$ be a set and $\SM$ be a $\sigma$-algebra on $X$. A measure on $\SM$, or $(X, \SM)$, is a function $\mu : \SM \to [0,\infty]$ such that \begin{enumerate}
        \item $\mu(\emptyset) = 0$,
        \item for a countable disjoint sequence $\{E_i\}_{i=1}^{\infty} \subset \SM$, it holds that $\mu\left(\cup_i E_i\right) = \sum_i \mu(E_i)$.
    \end{enumerate}
\end{definition}

\begin{definition}
    The set $(X, \SM)$ is called a measureable space, and the sets in $\SM$ are called measurable sets. The tuple $(X, \SM, \mu)$ is called a measure space. We say that $\mu$ is a finite measure if $\mu(X) < \infty$; finiteness implies that for all $E \in \SM$, $\mu(E) < \infty$. If $\mu(X) = 1$, then $\mu$ is called a probability measure.
\end{definition}

\begin{remark}
    Some important points.
    \begin{enumerate}
        \item We call property (ii) countable additivity.
        \item If $E_1, \ldots, E_n$ are disjoint sets in $\SM$, then $\mu\left(\cup_{i=1}^{n} E_i \right) = \sum_{i=1}^{n} \mu(E_i)$. This is called finite additivity, and is implied by countable additivity, since $E = E \cup \emptyset \cup \emptyset \cup \cdots$. 
        \item If $X = \bigcup_{j=1}^{\infty} E_j$ for $E_j \in \SM$ and $\mu(E_j) < \infty$ for all $j$, then $\mu$ is called $\sigma$-finite. For example, $\R$ is $\sigma$-finite.
        \item Suppose that for each $E \in \SM$ such that $\mu(E) = \infty$, there exists an $F \in \SM$ with $F \subset E$ and $0 < \mu(F) < \infty$. If $\mu$ satisfies this property, we call it semifinite.
    \end{enumerate}
\end{remark}

\begin{example}[Measures]
    \begin{itemize}
        \item Let $X$ be an infinite set and $\SM = 2^X$. Define \[\mu(E) = \begin{cases}
            0 &E \ \text{finite}\\
            \infty &E \ \text{infinite}.
        \end{cases}\] This is finitely additive, but not countably additive. So $\mu$ is a finitely additive measure but not a measure by our definition.
        \item Take $X$ to be any nonempty set, $\SM = 2^X$. Consider any function $f : X \to [0,\infty]$, and define \[\mu(E) = \sum_{x \in E} f(x) := \sup \left\{ \sum_{x \in F} f(x) : F \subset E, F \ \text{finite}\right\}.\] Also set $\mu(\emptyset) = 0$. Then $\mu$ is a measure. As an exercise, one should check that (i) $\mu$ is semifinite if and only if $f(x) < \infty$ for all $x \in X$, and (ii) $\mu$ is $\sigma$-finite if and only if $\mu$ is semifinite and $\{x : f(x) > 0\}$ is countable. \dhcomment{exercise, to-do.}
        \item Thinking more about the second example above, if we let $f(x) \equiv 1$, then $\mu$ is called the counting measure. If we have \[f(x) = \begin{cases}
            1 &x = x_0\\
            0 &x \neq x_0,
        \end{cases}\] then \[\mu(E) = \begin{cases}
            1 &x_0 \in E\\
            0 &\text{else}.
        \end{cases}\] This measure is called the point mass, or Dirac measure at $x_0$. 
    \end{itemize}
\end{example}

\begin{theorem}
    Let $(X, \SM, \mu)$ be a measure space. Then \begin{enumerate}
        \item (Monotonicity) If $E, F \in \SM$ and $E \subset F$, then $\mu(E) \leq \mu(F)$.
        \item (Subadditivity) If $\{E_j\}_{j=1}^{\infty} \subset \SM$, then \[\mu\left(\bigcup_{j=1}^{\infty} E_j \right) \leq \sum_{j=1}^{\infty} \mu(E_j).\]
        \item (Continuity from below) If $\{E_j\}_{j=1}^{\infty} \subset \SM$ and $E_1 \subset E_2 \subset \cdots$, then \[\mu\left(\bigcup_{j=1}^{\infty} E_j \right) = \lim_{j\to \infty} \mu(E_j).\]
        \item (Continuity from above) If $\{E_j\}_{j=1}^{\infty} \subset \SM$ and $E_1 \supset E_2 \supset \cdots$ and $\mu(E_1) < \infty$, then \[\mu\left(\bigcap_{j=1}^{\infty}E_j\right) = \lim_{j\to \infty} \mu(E_j).\]
    \end{enumerate}
\end{theorem}

\begin{proof}
    \begin{enumerate}
        \item Write $F = E \sqcup (F \setminus E)$. Then $\mu(F) = \mu(E) + \mu(F \setminus E) \geq \mu(E)$.
        \item Let $F_1 = E_1$, $F_k = E_k \setminus \bigcup_{j=1}^{k-1} E_j \subset E_k$. By construction, all $F_k$ are disjoint. Hence, \[\mu\left(\bigcup_{j=1}^{\infty} E_j \right) = \mu\left(\bigcup_{j=1}^{\infty} F_j \right) = \sum_{j=1}^{\infty} \mu(F_j) \leq \sum_{j=1}^{\infty} \mu(E_j).\]
        \item Suppose we have an increasing sequence $\{E_j\}_{j=1}^{\infty}$. Using a similar approach, and taking $E_0 = \emptyset$, we have \begin{align*}
            \mu\left(\bigcup_{j=1}^{\infty} E_j \right) = \mu\left(\bigcup_{j=1}^{\infty}\left(E_j \setminus E_{j-1}\right) \right) &= \sum_{j=1}^{\infty} \mu(E_j \setminus E_{j-1}) \\
            &= \lim_{n \to \infty} \left(\sum_{j=1}^{n} \mu(E_j \setminus E_{j-1})\right)\\
            &= \lim_{n\to \infty} \mu(E_n).
        \end{align*}
        \item Let $F_j = E_1 \setminus E_j$, so that we obtain an increasing sequence $F_1 \subset F_2 \subset \cdots$, and because $E_j \subset E_1$, we have $\mu(E_1) = \mu(F_j) + \mu(E_j)$. We have \[\bigcup_{j=1}^{\infty} F_j = E_1 \setminus \bigcap_{j=1}^{\infty} E_j, \ \ \ \text{and} \ \ \ E_1 = \left(\bigcap_{j=1}^{\infty} E_j\right) \sqcup \left(\bigcup_{j=1}^{\infty} F_j\right).\] That is, \[\mu(E_1) = \mu\left(\bigcap_{j=1}^{\infty} E_j \right) + \mu\left(\bigcup_{j=1}^{\infty} F_j \right),\] and by (iii), \[\mu\left(\bigcup_{j=1}^{\infty} F_j \right) = \lim_{n\to \infty} \mu(F_n) = \lim_{n \to 
        \infty} (\mu(E_1) - \mu(E_n)).\] Since $\mu(E_1) < \infty$, \[\mu\left(\bigcap_{j=1}^{\infty} E_j \right) = \lim_{j\to \infty} \mu(E_j).\]
    \end{enumerate}
\end{proof}

\begin{remark}
    It turns out that the assumption $\mu(E_1) < \infty$ is necessary in (iv). Consider the counting measure in $\R$ and $E_j = B_{1/j}(0)$. Then \[\bigcap_{j=1}^{\infty} E_j = \{0\}, \ \ \ \ \mu(\{0\}) = 1.\] but $\mu(E_j) = \infty$ for all $j$.
\end{remark}

\begin{definition}
    \begin{enumerate}
        \item If $(X, \SM, \mu)$ is a measure space, a set $E \in \SM$ with $\mu(E) = 0$ is called a null set. By subadditivity, a countable union of null sets is a null set.
        \item If a statement about points $x \in X$ is true except for the set $S$ such that $S \subset N$ for a null set $N$, we say the statement is true almost everywhere (or a.e.). To be more specific, we call $N$ a $\mu$-null set or say that the statement holds $\mu$-almost-everywhere.
    \end{enumerate}
\end{definition}

\begin{example}
    $f(x) = 1/x$ is defined a.e. on $\R$ with respect to the Lebesgue measure.
\end{example}

\begin{definition}
    A measure whose domain includes all subset of null sets is called complete.
\end{definition}

\begin{theorem}
    Let $(X, \SM, \mu)$ be a measure space. Let \[\CN = \{N \in \SM : \mu(N) = 0\}, \ \ \ \ol{\SM} = \{E \cup F : E \in \SM, F \subset N \in \CN\}.\] Then $\ol{\SM}$ is a $\sigma$-algebra, and there exists a unique extension $\ol{\mu}$ of $\mu$ to a complete measure on $\ol{\SM}$. We call $\ol{\mu}$ the completion of $\mu$.
\end{theorem}

\begin{proof}
    Closure under countable union is straightforward. To see complements, let $E \cup F \in \ol{\SM}$, for $F \subset N$. We may assume $E \cap F = \emptyset$, just by taking the set difference. So we have $E \cap N^c = E$ and $N \cap F = F$. Then \[E \cup F = (E \cup N) \cap (N^c \cup F).\] So \[(E \cup F)^c = (E \cup N)^c \cup (N \setminus F),\] and this lies in $\ol{\SM}$.

    Choose any $E \cup F \in \ol{\SM}$. We define the extension $\ol{\mu}$ to be \[\ol{\mu}(E \cup F) = \mu(E), \ \ \ \text{for each $E \cup F \in \ol{\SM}$}.\] Of course, the representation for the set $E \cup F$ is not unique, so we need to check if this is well-defined. Suppose $E_1 \cup F_1 = E_2 \cup F_2 \in \ol{\SM}$, $F_1 \subset N_1$ and $F_2 \subset N_2$. Clearly $E_1 \subset E_2 \cup F_2 \subset E_2 \cup N_2$. So \[\mu(E_1) \leq \mu(E_2) + \mu(N_2) = \mu(E_2).\] This goes in both directions, so $\ol{\mu}$ is well-defined. It remains to show that $\ol{\mu}$ is complete and unique (Homework 2).
\end{proof}

\subsection{Outer Measures}

We are interested in finding a practical way to construct measures. We also want some approximate way to measure sets that is easy to use. We address both desires with the notion of an outer measure.

\begin{definition}[Outer measure]
    An outer measure on a nonempty set $X$ is a function $\mu^* : 2^X \to [0,\infty]$ such that \begin{enumerate}
        \item $\mu^*(\emptyset) = 0$,
        \item $\mu^*(A) \leq \mu^*(B)$ if $A \subset B$,
        \item $\mu^*\left(\bigcup_{j=1}^{\infty} A_j\right) \leq \sum_{j=1}^{\infty} \mu^*(A_j)$.
    \end{enumerate}
\end{definition}

\begin{proposition}
    Let $\CE \subset 2^X$ and $\vr : \CE \to [0, \infty]$ be such that (i) $\emptyset \in \CE$, (ii) $X \in \CE$, (iii) $\vr(\emptyset) = 0$. For any subset $A \subset X$, define \[\mu^*(A) := \inf \left\{\sum_{j=1}^{\infty}\vr(E_j) : E_j \in \CE, A \subset \bigcup_{j=1}^{\infty}E_j\right\}.\] Then $\mu^*$ is an outer measure.
\end{proposition}

\begin{proof}
    First note that for any $A \subset X$, there exists a countable sequence $\{E_j\}_{j=1}^{\infty} \subset \CE$ such that $A \subset \bigcup E_j$ because $X \in \CE$. So the infemum exists and $\mu^*$ is well-defined. (1) Note that $\mu^*(\emptyset) = 0$ since we can cover $\emptyset$ by taking $E_j = \emptyset$ for all $j$. (2) Let $A \subset B$. Then any collection $E_j \in \CE$ with $B \subset \bigcup E_j$ also satisfies $A \subset \bigcup E_j$, so $\mu^*(A) \leq \mu^*(B)$. (3) Let $\{A_j\}_{j=1}^{\infty} \subset 2^X$ and $\eps > 0$. For each $j$ there exists $\{E^j_k\}_{k=1}^{\infty} \subset \CE$ such that $A_j \subset \bigcup_k E^j_k$ and $\sum_{k} \vr(E_k^j) \leq \mu^*(A_j) + \eps{2^{-j}}$. It follows that $\bigcup_j A_j \subset \bigcup_{j,k} E^j_k$, and \[\mu^*\left(\bigcup_j A_j\right) \leq \sum_{j, k} \vr(E^j_k) \leq \sum_j \mu^*(A_j) + \eps.\] Since $\eps$ is arbitrary, this completes the proof.
\end{proof}


Given a function $\vr: \CE \to [0,\infty]$, we have seen that we can construct an outer measure $\mu^* : 2^X \to [0,\infty]$. Now, starting with the outer measure, we want to come up with a measure $\mu$. We start by finding a $\sigma$-algebra $\SA$ such that the restriction $\mu = \restr{\mu^*}{\SA} : \SA \to [0,\infty]$ forms a measure.

\begin{definition}[$\mu^*$-measurability]
    Given an outer measure $\mu^* : 2^X \to [0,\infty]$, we call $A \subset X$ a $\mu^*$-measurable set if it holds that for all $E \subset X$, \[\mu^*(E) = \mu^*(E \cap A) + \mu^*(E \cap A^c).\]
\end{definition}

\begin{remark}
    We spend some time thinking about how to show $\mu^*$-measurability and what the implications of this notion are.
    \begin{itemize}
        \item For any $A$ and $E$, since $E \subset (E \cap A) \cup (E \cap A^c)$, properties of the outer measure give that \[\mu^*(E) \leq \mu^*(E \cap A) + \mu^*(E \cap A^c).\] It therefore suffices to show the reverse inequality to prove that $A$ is $\mu^*$-measrable. If $\mu^*(E) = \infty$, then the reverse inequality is trivial. Hence, we see that $A$ is $\mu^*$-measurable if and only if \[\mu^*(E) \geq \mu^*(E \cap A) + \mu^*(E \cap A^c)\] for all $E \subset X$ with $\mu^*(E) < \infty$.
        \item Suppose $A$ is $\mu^*$-measurable, and say $A \subset E$. Then $E \cap A$ and $E \cap A^c$ form a disjoint partition of $E$. In particular we have that \[\mu^*(E) = \mu^*(E \cap A) + \mu^*(E \cap A^c) = \mu^*(A) + \mu^*(E \cap A^c).\] That is, $\mu^*(A) = \mu^*(E) - \mu^*(E  \cap A^c)$. 
        \item Supppose $A, B$ are disjoint and $A$ is $\mu^*$-measurable. Then by taking $E = A \sqcup B$, we have \[\mu^*(A \cup B) = \mu^*((A \cup B) \cap A) + \mu^*((A \cup B) \cap A^c) = \mu^*(A) + \mu^*(B).\]
    \end{itemize}
\end{remark}

\begin{theorem}[Caratheodory]
    Let $\mu^*$ be an outer measure on $X$. The collection $\SM$ of $\mu^*$-measurable sets is a $\sigma$-algebra, and the restriction $\restr{\mu^*}{\SM}$ is a complete measure on $\SM$.
\end{theorem}

\begin{proof}
    As a first step, we show $\SM$ is a $\sigma$-algebra. The definition of $\mu^*$-measurable is symmetric in $A$, $A^c$, so $A \in \SM$ implies $A^c \in \SM$. To show closure under finite union, take $A, B \in \SM$ and $E \subset X$. Then \begin{align*}
        \mu^*(E) &= \mu^*(E \cap A) + \mu^*(E \cap A^c)\\
        &= \mu^*(E \cap A \cap B) + \mu^*(E \cap A \cap B^c) + \mu^*(E \cap A^c \cap B) + \mu^*(E \cap A^c \cap B^c)\\
        &\geq \mu^*(E \cap (A \cup B)) + \mu^*(E \cap A^c \cap B^c),
    \end{align*} where the last step follows because $A \cup B = (A \cap B) \cup (A \cap B^c) \cup (A^c \cap B)$. Therefore, \[\mu^*(E) \geq \mu^*(E \cap (A \cup B)) + \mu^*(E \cap (A \cup B)^c),\] and $A \cup B \in \SM$.

    The second step is to show that $\SM$ is a $\sigma$-algebra. By our previous arguments, it's sufficient to show $\SM$ is closed under disjoint countable unions. Let $\{A_j\}_{j=1}^{\infty}$ be a countable sequence of disjoint sets in $\SM$ and define \[B_n = \bigcup_{j=1}^{n} A_j,\] so $\{B_n\}_{n=1}^{\infty}$ is increasing. Moreover, $B := \bigcup_{j=1}^{\infty} A_j = \bigcup_{n=1}^{\infty} B_n$. Consider a test set $E \subset X$. Since $A_j$ is $\mu^*$-measurable, \begin{align*}
        \mu^*(E \cap B_n) &= \mu^*(E \cap B_n \cap A_n) + \mu^*(E \cap B_n \cap A_n^c)\\
        &= \mu^*(E \cap A_n) + \mu^*(E \cap B_{n-1})\\
        &= \mu^*(E \cap A_n) + \mu^*(E \cap A_{n-1}) + \mu^*(E \cap B_{n-2})\\
        &= \sum_{j=1}^{n} \mu^*(E \cap A_j).
    \end{align*} Next, we note that \begin{align*}
        \mu^*(E) &= \mu^*(E \cap B_n) + \mu^*(E \cap B_n^c)\\
        &= \sum_{j=1}^{n} \mu^*(E \cap A_j) + \mu^*(E \cap B_n^c)\\
        &\geq \sum_{j=1}^{n} \mu^*(E \cap A_j) + \mu^*(E \cap B^c).
    \end{align*} Since $n$ was arbitrary, we can take the limit on the RHS to obtain \begin{align*}
        \mu^*(E) &\geq \sum_{j=1}^{\infty} \mu^*(E \cap A_j) + \mu^*(E \cap B^c)\\
        &\geq \mu^*\left(\bigcup_{j=1}^{\infty} E \cap A_j \right) + \mu^*(E \cap B^c)\\
        &= \mu^*(E \cap B) + \mu^*(E \cap B^c) \geq \mu^*(E).
    \end{align*}
    The intermediate inequalities are therefore equalities, and we have $B = \bigcup_{j=1}^{\infty} A_j \in \SM$.

    The third step is to show that $\restr{\mu^*}{\SM}$ is a complete measure. In the previous step, take $E = B$. Then \[\mu^*(B) = \mu^*\left(\bigcup_j A_j \right)= \sum_{j=1}^{\infty} \mu^*(A_j).\] That is, $\mu^*$ is countably additive, and a measure. It remains to show completeness. Let $A \subset X$ with $\mu^*(A) = 0$. For any $E \subset X$, monotonicity implies \[\mu^*(E) \leq \mu^*(E \cap A) + \mu^*(E \cap A^c) = \mu^*(E \cap A^c) \leq \mu^*(E).\] All inequalities must be equalities, so $A \in \SM$. That is, any set whose outer measure is zero is included in $\SM$. That is, $\SM$ includes all possible null sets and $\restr{\mu^*}{\SM}$ is a complete measure.
\end{proof}

\subsection{Premeasures}

\begin{definition}
    Let $\SA \subset 2^X$ be an algebra (not nec. a $\sigma$-algebra). Let $\mu_0 : \SA \to [0,\infty]$ satisfy \begin{enumerate}
        \item $\mu_0(\emptyset) = 0$,
        \item for a sequence $\{A_i\}_{i=1}^{\infty}$ of disjoint sets and $\bigcup_i A_i \in \SA$, then $\mu_0\left(\bigcup_i A_i \right) = \sum_i \mu_0(A_i)$.
    \end{enumerate}
    If both of these conditions are satisfied, we call $\mu_0$ a premeasure.
\end{definition}

\begin{remark}
    Every premeasure is finitely additive. Take $A_i = \emptyset$ for $i$ large.
\end{remark}

\begin{proposition}
    If $\mu_0$ is a premeasure on an algebra $\SA$. Define the outer measure \[\mu^*(E) = \inf \left\{\sum_{j=1}^{\infty} \mu_0(A_j) : A_j \in \SA, E \subset \bigcup_{j=1}^{\infty} A_j \right\}.\] Then the following hold. \begin{enumerate}
        \item $\restr{\mu^*}{\SA} = \mu_0$;
        \item every set in $\SA$ is $\mu^*$-measurable.
    \end{enumerate}
\end{proposition}

\begin{proof}
    To see the first point, let $E \in \SA$. Then, for an arbitrary cover $\{A_j\}$, we have $E \subset \bigcup_j A_j$. We define \[B_n = E \cap \left(A_n \setminus \bigcup_{j=1}^{n-1} A_{j} \right) \in \SA.\] Note that $B_n$'s are disjoint, $\bigcup_n B_n = E$, and $B_n \subset A_n$. Since $\mu_0$ is a premeasure, \[\mu_0(E) = \sum_{n=1}^{\infty} \mu_0(B_n) \leq \sum_{n=1}^{\infty} \mu_0(A_n),\] where the last inequality uses monotonicity, which can be shown for premeasures. That is, $\mu_0(E) \leq \mu^*(E)$. In order to see the other inequality, consider the covering $E \subset A_1 \in \SA$, for $A_1 := E$. We immediately see $\mu^*(E) \leq \mu_0(E)$.

    To see the second point, let $A \in \SA$ and $E \subset X$. Take arbitrary $\eps > 0$. By the definition of $\mu^*(E)$, there exists a sequence $\{B_j\}_{j=1}^{\infty} \subset \SA$ such that $E \subset \bigcup_j B_j$ and \[\sum_{j=1}^{\infty} \mu_0(B_j) \leq \mu^*(E) + \eps.\] Since $\mu_0$ is finitely additive, \[\mu_0(B_j) = \mu_0(B_j \cap A) + \mu_0(B_j \cap A^c).\] Now, we have \begin{align*}
        \mu^*(E) + \eps \geq \sum_{j=1}{^\infty} \mu_0(B_j) &= \sum_{j=1}^{\infty} \mu_0(B_j \cap A) + \sum_{j=1}^{\infty} \mu_0(B_j \cap A^c).
    \end{align*}
    We notice that the left and right terms introduce covers for $E \cap A$ and $E \cap A^c$, respectively. Hence, we have \[\mu^*(E) + \eps \geq \mu^*(E \cap A) + \mu^*(E \cap A^c).\] Since $\eps > 0$ is arbitrary, we have one inequality. The other is clear from the definition of outer measure.
\end{proof}

\begin{theorem}\label{thm:premeasure-induced-measure}
    Let $\SA \subset 2^X$ be an algebra. Let $\mu_0$ be a premeasure on $\SA$. Consider the generated $\sigma$-algebra $\SM = \SM(\SA)$. There exists a measure $\mu$ in $\SM$ such that $\restr{\mu}{\SA} = \mu_0$. If $\nu$ is another measure on $\SM$ that extends $\mu_0$, then $\nu(E) \leq \mu(E)$ for all $E \in \SM$ with equality when $\mu(E) < \infty$. If $\mu_0$ is $\sigma$-finite, then $\restr{\mu^*}{\SM}$ is the unique extension of $\mu_0$ to a measure on $\SM$. 
\end{theorem}

\begin{proof}
    The first assertion follows from the previous proposition and Caratheodory's Theorem. Take $\mu^* : 2^X \to [0, \infty]$ to be the canonical outer measure, which restricts to $\mu_0$ on $\SA$. Moreover, by the previous proposition, $\SA$ is contained in the set $\SC$ of $\mu^*$-measurable sets. By Caretheodory's theorem, $\SC$ is a $\sigma$-algebra and $\restr{\mu^*}{\SC}$ is a complete measure on $\SC$. Hence, $\SM = \SM(\SA) \subset \SC$, and $\restr{\mu^*}{\SM}$ is a measure on $\SM$, which restricts to $\mu_0$ on $\SA$.
    
    We now focus on the second claim. Let $E \in \SM$ and consider a cover $E \subset \bigcup_{j=1}^{\infty} A_j$ for $A_j \in \SA$. Then \[\nu(E) \leq \sum_{j=1}^{\infty} \nu(A_j) = \sum_{j=1}^{\infty} \mu_0(A_j).\] Since $\{A_j\}_j$ was an arbitrary cover, this gives $\nu(E) \leq \mu(E)$. 
    
    Now we make an interesting observation. If we set $A = \bigcup_{j=1}^{\infty} A_j$, then by continuity from below and consistency on $\CA$, \[\nu(A) = \lim_{n\to \infty} \nu\left(\bigcup_{j=1}^{N} A_j\right) = \lim_{n \to \infty} \mu\left(\bigcup_{j=1}^{N} A_j \right) = \mu(A).\] This observation becomes useful. If $\mu(E) < \infty$\footnote{Note how we use finiteness of $\mu(E)$ here.}, we can choose $A_j$'s so that $\mu(A) < \mu(E) + \eps$, and hence $\mu(A \setminus E) < \eps$ and \[\mu(E) \leq \mu(A) = \nu(A) = \nu(E) + \nu(A \setminus E) \leq \nu(E) + \mu(A \setminus E) \leq \nu(E) + \eps.\] Since $\eps > 0$ was arbitrary, we have $\mu(E) = \nu(E)$.

    Finally, suppose $X = \bigcup_{j=1}^{\infty} A_j$ with $\mu_0(A_j) < \infty$, where we can assume the $A_j$'s are disjoint. Then for any $E \in \SM$, we have \[\mu(E) = \sum_{j=1}^{\infty} \mu(E \cap A_j) = \sum_{j=1}^{\infty} \nu(E \cap A_j) = \nu(E),\] so $\nu = \mu$.
\end{proof}

\begin{remark}
    On sets where $\mu(E) = \infty$, then the extension may not be unique. For instance, let $\SA$ be generated by $(-\infty, 0]$ and all intervals of the form $(a, b)$ with $0 < a, b$. Let $\mu_0((a, b)) = b-a$. One can check that this is a premeasure. Then let $\nu((c, d)) = \int_c^d \vp(x) \d x$ with $\vp \in C(\R)$, $\vp\geq 0$ and $\vp(x) = 1$ for $x \geq 0$ and $\int_{-\infty}^{\infty} \vp(x) \d x = 0$. One can check that no matter what $\vp$ is, this is an extension. \dhcomment{Check as exercise.}
\end{remark}

\subsection{Borel Measures on $\R$}

We will now use some of the tools we have to construct important measures. Specifically, our goal will be to construct a measure $\mu$ such that $\mu((a, b)) = b - a$ on $\SB_\R$. Later on, we will call one such measure the Lebesgue measure, but for now we consider a more general family of measures whose domain is $\SB_\R$. 

% We find some motivation in probability, and this motivation will help us in the construction. Given a finite Borel measure on $\mu$, we call a function $F(x) = \mu((-\infty, x])$ a distribution function of $\mu$. Since $(-\infty, x] = \bigcap (-\infty, x_n)$ for $x_n \searrow x$, we have right continuity of $F$. Also, if $b > a$, then $(-\infty, b] = (-\infty, a] \cup (a, b]$, so $\mu((a, b]) = F(b) - F(a)$. We will construct $\mu$ starting from an increasing, right-continuous function $F$.

\begin{definition}[Borel measure]
    For a measure $\mu : \SB_\R \to [0,\infty]$, we call $\mu$ a Borel measure.
\end{definition}

\begin{definition}[$h$-interval]
    Consider all sets of the form $(a, b]$ or $(a, \infty)$ or $\emptyset$ where $-\infty \leq a < b < \infty$. We refer to such sets as $h$-intervals.
\end{definition}

\begin{remark}
    The intersection of two $h$-intervals is an $h$-interval. Complements of $h$-intervals in $\R$ are either $h$-intervals, or a disjoint union of two $h$-intervals. The collection of $h$-intervals forms an elementary family. The collection $\SA$ of finite disjoint unions of $h$-intervals is an algebra, and $\SM(\SA) = \SB_\R$.
\end{remark}

\begin{proposition}
    Let $F : \R \to \R$ be increasing and right-continuous. If $(a_j, b_j]$ for $j \in \N$ are disjoint $h$-intervals, define the function $\mu_0$ with the expression \[\mu_0 \left(\bigcup_1^n (a_j, b_j] \right) = \sum_{1}^{n} F(b_j) - F(a_j),\] and let $\mu_0(\emptyset) = 0$. We claim that $\mu_0$ is well-defined and a premeasure on the algebra $\SA$.
\end{proposition}

\begin{proof}
    First suppose that $\bigcup_1^n (a_j, b_j] = (a, b]$, then we must have, after relabeling index $j$, \[a = a_1 < b_1 = a_2 < b_2 = \cdots < b_n = b.\] Hence, \[\mu_0 \left( \bigcup_1^n (a_j, b_j] \right) = \sum_1^n F(b_j) - F(a_j) = F(b) - F(a) = \mu_0((a, b]).\]
    Now, more generally, if $\bigcup_1^n I_i = \bigcup_1^m J_j$ are two disjoint unions of $h$-intervals, we can partition both further into \[\wt{I}_{ij} = I_i \cap J_j,\] so that \[\sum_i \mu_o(I_i) = \sum_{i, j} \mu_0(\wt{I}_{ij}) = \sum_j \mu_0(J_j).\] So $\mu_0$ is well-defined, and we note that by construction, $\mu_0$ is also finitely additive.

    We now aim to show that $\mu_0$ is a premeasure. The only remaining property to prove is: if $\{I_j\}_{j=1}^{\infty}$ is a countable disjoint sequence in $\SA$ with $\bigcup_1^\infty I_j \in \SA$, then $\mu_0\left(\bigcup_1^\infty I_j\right) = \sum_1^\infty \mu_0(I_j)$. By definition of $\SA$, there is a finite disjoint collection of $h$-intervals $\{J_{i}\}_{i=1}^{m}$ such that $\bigcup I_j = \bigcup_{i=1}^{m} J_i$. In particular, we can partition $\{I_j\}$ into finitely many subsequences of the form $\{J_i \cap I_j\}_{j=1}^{\infty}$ such that the union of the intervals in each subsequence is a single $h$-interval $J_i$. Considering each subsequence separately and using the finite additivity of $\mu_0$, we may assume that $\bigcup_1^\infty I_j$ is an $h$-interval $I = (a, b]$. In this case, we have that for any $n \in \N$, \[\mu_0(I) = \mu_0\left( \bigcup_1^n I_j \right) + \mu_0 \left( I \setminus \bigcup_1^n I_j \right) \geq \mu_0\left(\bigcup_1^n I_j \right) = \sum_1^n \mu_0(I_j).\] Letting $n \to \infty$, we have $\mu_0(I) \geq \sum_1^\infty \mu(I_j)$.

    To prove the reverse inequality, first suppose that $a, b$ finite, and let $\eps > 0$. Since $F$ is right-continuous, there exists $\delta > 0$ such that $F(a + \delta) - F(a) < \eps$, and if $I_j = (a, b_j]$, then for each $j$ there exists $\delta_j > 0$ such that $F(b_j + \delta_j) - F(b_j) < \eps2^{-j}$. The open intervals cover the compact set $[a + \delta, b]$, so there is a finite subcover. By discarding any $(a_j, b_j + \delta_j)$ that is contained in a larger one and relabeling the index $j$, we may assume two additional conditions: \begin{enumerate}
        \item the intervals $(a_1, b_1 + \delta_1), \ldots, (a_N, b_N + \delta_N)$ cover $[a + \delta, b]$,
        \item $b_j + \delta_j \in (a_{j+1}, b_{j+1} + \delta_{j+1})$ for $j = 1, \ldots, N-1$.
    \end{enumerate}
    It follows that \begin{align*}
        \mu_0(I) &\leq F(b) - F(a + \delta) + \eps\\
        &\leq F(b_N + \delta_N) - F(a_1) + \eps\\
        &= F(b_N + \delta_N) - F(a_N) + \sum_1^{N-1} \Big( F(a_{j+1}) - F(a_j) \Big) + \eps\\
        &\leq F(b_N + \delta_N) - F(a_N) + \sum_1^{N-1} \Big( F(b_{j} + \delta_j) - F(a_j) \Big) + \eps\\
        &< \sum_1^N \Big( F(b_j) + \eps 2^{-j} - F(a_j) \Big) + \eps < \sum_1^\infty \mu_0(I_j) + 2 \eps.
    \end{align*}
    Since $\eps > 0$ is arbitrary, this concludes the proof for the case where $a$ and $b$ are finite. When $a = -\infty$, for any $M < \infty$, the intervals $(a_j, b_j + \delta_j)$ cover $[-M, b]$, so the same reasoning gives $F(b) - F(-M) \leq \sum_1^\infty \mu_0(I_j) + 2\eps$. If $b = \infty$, then for any $M < \infty$ we likewise obtain $F(M) - F(a) \leq \sum_1^\infty \mu_0(I_j) + 2\eps$. The desired result then follows by letting $\eps \to 0$ and $M \to \infty$.
\end{proof}

\begin{theorem}\label{thm:borel-measure-correspondence}
    If $F : \R \to \R$ is an increasing, right continuous function, there is a unique Borel measure $\mu_F$ on $\R$ such that $\mu_F((a, b]) = F(b) - F(a)$ for all $a, b$. If $G$ is another such function, we have $\mu_F = \mu_G$ if and only if $F - G$ is constant. Conversely, if $\mu$ is a Borel measure on $\R$ that is finite on all bounded Borel sets and we define \[F(x) = \begin{cases}
        \mu((0, x]) &x > 0\\
        0 &x = 0\\
        -\mu((-x, 0]) &x < 0,
    \end{cases}
    \] then $F$ is increasing and right continuous and $\mu = \mu_F$.
\end{theorem}

\begin{proof}
    To start, by the previous proposition, each $F$ induces a premeasure on on the algebra $\SA$ of finite disjoint unions of $h$-intervals. Also, it is clear that $F$ and $G$ induce the same premeasure if and only if $F - G$ is contant, and that these premeasures are $\sigma$-finite. The first two assertions therefore follow from \cref{thm:premeasure-induced-measure}. 

    As for the last claim, assume $\mu$ is a Borel measure that is finite on all Borel sets. The monotonicity of $\mu$ implies the monotonicity of $F$, and the continuity of $\mu$ from above and below implies right continuity of $F$ for $x \geq 0$ and $x < 0$. Now on $\SA$, it is clear that $\mu = \mu_F$, and hence $\sigma$-finiteness and the uniqueness result in \cref{thm:premeasure-induced-measure} imply that $\mu = \mu_F$ on $\SB_\R$.
\end{proof}

\begin{remark}
    A few points regarding the theory developed in this subsection.
    \begin{enumerate}
        \item The theory could be equally well be developed using intervals of the form $[a, b)$ and left-continuous functions $F$.
        \item If $\mu$ is a finite Borel measure on $\R$, then $\mu = \mu_F$ where $F(x) = \mu((-\infty, x])$, and we call $F$ the cumulative distribution function of $\mu$. This differs from the function defined in \cref{thm:borel-measure-correspondence} by the constant $\mu((-\infty, 0])$.
        \item By the Caratheodory theorem, each increasing right-continuous function $F$ gives us, not only a Borel measure $\mu_F$, but a complete measure $\ol{\mu}_F$ whose domain includes $\SB_\R$, where $\ol{\mu}_F$ is the completion of $\mu_F$ \citep[Exercise~22a]{folland1999real}. One can show that the domain will always be strictly larger that of $\SB_\R$. We call $\ol{\mu}_F$ the Lebesgue-Stieltjes measure associated to $F$.
    \end{enumerate}
\end{remark}

\subsubsection{Lebesgue-Stieltjes Measure}

We now investigate the properties of the Lebesgue-Stieltijes measures. For the remainder of the subsection, fix a complete Lebesgue-Stieltjes measure $\mu$ associated to the increasing right-continuous function $F$ and denote the domain $\sigma$-algebra of $\mu$ by $\SM_\mu$.

\begin{lemma}
    For any $E \in \SM_\mu$, then \begin{align*}
        \mu(E) &= \inf \left\{ \sum_1^\infty \mu((a_j, b_j)) : E \subset \bigcup_1^\infty (a_j, b_j) \right\}.
    \end{align*}
\end{lemma}

\begin{proof}
    By construction of the complete measure, \begin{align*}
        \mu(E) &= \inf \left\{ \sum_1^\infty \Big(F(b_j) - F(a_j) \Big) : E \subset \bigcup_1^\infty (a_j, b_j] \right\}\\
        &= \inf \left\{ \sum_1^\infty \mu((a_j, b_j]) : E \subset \bigcup_1^\infty (a_j, b_j] \right\}.
    \end{align*}
    Now define \[\nu(E) = \inf \left\{ \sum_1^\infty \mu((a_j, b_j)) : E \subset \bigcup_1^\infty (a_j, b_j) \right\}.\] We want to show $\mu(E) = \nu(E)$. To see that $\mu(E) \leq \nu(E)$, start by supposing that $E \subset \bigcup_1^{\infty} (a_j, b_j)$. Each $(a_j, b_j)$ can be written as a countable disjoint union of $h$ intervals $\{I_j^k\}_k$. Specifically, $I_j^k = (c_j^k, c_j^{k+1}]$ where $\{c_j\}_j$ is any sequence such that $c_j^1 = a_j$ and $c_j^k$ increases to $b_j$ as $k \to \infty$. Thus, \[E \subset \bigcup_{j, k} I_j^k.\] In particular, we have \[\sum_1^{\infty} \mu((a_j, b_j)) = \sum_{j, k} \mu(I_j^k) \geq \mu(E).\] Since this covering was arbitrary, $\nu(E) \geq \mu(E)$.

    On the other hand, given $\eps > 0$, there exists $\{(a_j, b_j]\}_1^{\infty}$ where $E \subset \bigcup_1^{\infty} (a_j, b_j]$ and $\sum_1^{\infty} \mu((a_j, b_j]) \leq \mu(E) + \eps$. Moreover, for each $j$, there exists some $\delta_j > 0$ such that $F(b_j + \delta_j) - F(b_j) < \eps 2^{-j}$. Then $E \subset \bigcup_1^{\infty} (a_j, b_j + \delta_j)$ and \[\sum_1^\infty \mu((a_j, b_j + \delta_j)) \leq \sum_1^\infty \mu((a_j, b_j]) + \eps \leq \mu(E) + 2\eps.\] Since the cover was arbitrary, $\nu(E) \leq \mu(E)$.
\end{proof}

\begin{theorem}
    Let $E \in \SM_\mu$. Then \begin{align*}
        \mu(E) &= \inf\{\mu(U) : U \supset E \ \text{and $U$ is open}\}\\
        &= \sup\{\mu(K) : K \subset E \ \text{and $K$ is compact}\}.
    \end{align*}
\end{theorem}

\begin{proof}
    By the previous lemma, for any $\eps > 0$, there exists intervals $(a_j, b_j)$ such that $E \subset \bigcup_1^\infty (a_j, b_j)$ and $\mu(E) \leq \sum_1^\infty \mu((a_j, b_j)) + \eps$. If we define $U = \bigcup_1^\infty (a_j, b_j)$, then $U$ is open, $U \supset E$, and $\mu(U) \leq \mu(E) + \eps$. On the other hand, $\mu(U) \geq \mu(E)$ whenever $U \supset E$, so the first equality is valid.

    For the second equality, first suppose that $E$ is bounded. If $E$ is closed, then $E$ is compact and the equality is obvious. Otherwise, given $\eps > 0$, we can choose an open set $U \supset \ol{E} \setminus E$ such that $\mu(U) \leq \mu(\ol{E} \setminus E) + \eps$, by the earlier paragraph. Let $K = \ol{E} \setminus U$, so $K$ compact and $K \subset E$. Also, \begin{align*}
        \mu(K) &= \mu(E) - \mu(E \cap U) = \mu(E) - [\mu(U) - \mu(U\setminus E)]\\
        &\geq \mu(E) - \mu(U) + \mu(\ol{E}\setminus E) \geq \mu(E) - \eps.
    \end{align*}
    If $E$ is unbounded, let $E_i = E \cap (j, j+1]$. The the preceding argument, for any $\eps > 0$, there exists compact $K_j \subset E_j$ such that $\mu(K_j) \geq \mu(E_j) - \eps2^{-j}$. Let $H_n = \bigcup_{-n}^{n} K_j$, so $H_n$ is compact, $H_n \subset E$, and for any $n \in \N$, $\mu(H_n) \geq \mu(\bigcup_{-n}^{n} E_j) - \eps$. Since $\mu(E) = \lim_{n \to \infty} \mu(\bigcup_{-n}^{n}E_j)$, the result follows.
\end{proof}

\begin{theorem}
    If $E \subset \R$, then the following are equivalent.\begin{enumerate}
        \item $E \in \SM_\mu$,
        \item $E = V \setminus N_1$ where $V$ is a $G_\delta$ set and $\mu(N_1) = 0$,
        \item $E = H \cup N_2$ where $H$ is an $F_\sigma$ set and $\mu(N_2) = 0$.
    \end{enumerate}
\end{theorem}

\begin{proof}
    It's clear that (b) and (c) imply (a) since $\mu$ is complete on $\SM_\mu$. Suppose that $E \in \SM_\mu$ and $\mu(E) < \infty$. By the theorem above, for $j \in \N$, we can choose an open $U_j \supset E$ and compact $K_j \subset E$ such that \[\mu(U_j) - 2^{-j} \leq \mu(E) \leq \mu(K_j) + 2^{-j}.\]
    Let $V = \bigcap_1^\infty U_j$ and $H = \bigcup_1^\infty K_j$. Then $H \subset E \subset V$ and $\mu(V) = \mu(H) = \mu(E) < \infty$. Hence, $\mu(V \setminus E) = \mu(E \setminus H) = 0$. The result is thus proved when $\mu(E) < \infty$; an extension to the general case is left to \citet[Exercise~25]{folland1999real}.
\end{proof}

\begin{remark}
    This theorem is really saying that $\SM_\mu$ consists of Borel sets modulo null sets.
\end{remark}

\subsubsection{The Lebesgue Measure}

We now turn to a special case of the complete measure $\mu_F$, associated with the identity $F(x) = x$. We call this the Lebesgue measure $m$ with domain $\SL$. A set $L \in \SL$ is called Lebesgue measurable. In a slight abuse of terminology, we sometimes call the restriction $\restr{m}{\SB_\R}$ the Lebesgue measure, though this will always be specified or clear from context. We will see that this measure space satisfies very nice properties.

\begin{theorem}
    Let $E \in \SL$. Then $E + s \in \SL$ and $rE \in \SL$ for all $r, s \in \R$, where \[E + s = \{x + s : x \in S\}, \ \ \ \ rE = \{r \cdot x : x \in E\}.\] Moreover, $m(E + s) = m(E)$ and $m(rE) = |r| \cdot m(E)$.
\end{theorem}

\begin{proof}
    Since the collection of open intervals is invariant under translations and dialations, so is $\SB_\R$. For $E \in \SB_\R$, define \[m_s(E) = m(E + s), \ \ \ m^r(E) = m(rE).\] Notice that for finite unions of intervals, $m_s = m$ and $m^r = |r| \cdot m$. Since these measures agree on the algebra of finite unions of intervals, they agree on $\SB_\R$ by \cref{thm:premeasure-induced-measure}. In particular, all null sets in $\SB_\R$ are preserved under scaling and transformation. By the characterization of elements of $\SL$ as a union of Borel sets and a Lebesgue null set, measures on $\SL$ are preserved under translations and dialations.
\end{proof}

\begin{remark}
    \begin{enumerate}
        \item As seen on homework, the Lebesgue measure of $\Q$ is zero. More generally, if the set $E$ is countable, $m(E) = 0$.
        \item Let $\{q_j\}$ be an enumeration of $\Q \cap [0,1]$ and $\bigcup_i B_{r_j}(q_j)$ where $r_j = \eps / 2^j$. Now take the intersection $U = (0,1) \bigcap \left( \bigcup_i B_{r_j}(q_j) \right)$. So $U$ is open and dense in $[0,1]$. Note, however, $m(U) \leq \sum_1^\infty \eps 2^{-j+1} = \eps$. Moreover, the set $K = [0,1] \setminus U$ is closed, nowhere dense\footnote{The closure has empty interior.}, but $m(K) \geq 1 - \eps$. The point here is that a ``topologically large'' set can have a very small measure and a ``topologically small'' set can have a very large measure.
        \item Any nonempty open set has a positive Lebesgue measure. Indeed, nonempty open sets will contain some nontrivial open interval by definition, and open intervals have positive measure.
        \item The Lebesgue null sets include not only all countable sets, but also some sets with cardinality of the continuum. We will address this fact below.
    \end{enumerate}
\end{remark}

\subsubsection{Null Sets Under Lebesgue Measure}

We start with the Cantor set.

\begin{definition}[Cantor set, Cantor function]
    Define $C$ as the set of all $x \in [0,1]$ that have a base-3 expansion \[x = \sum_j a_j \cdot 3^{-j} \ \ \ \ \text{with} \ a_j \in \{0,2\} \ \ \forall j \in \N.\]
    The set $C$ can be obtained by removing the open middle third $(1/3, 2/3)$ from $[0,1]$, then the open middle thirds of the remaining intervals $(1/9, 2/9)$ and $(7/9, 8/9)$, etc. 
    
    We also define the Cantor function $f : C \to [0,1]$ mapping \[\sum a_j 3^{-j} \mapsto \sum \left(\frac{a_j}{2}\right) 2^{-j}.\] In the image, we have a base-2 expansion $f(x) = \sum_1^\infty b_j 2^{-j}$ where $b_j = a_j / 2$.
\end{definition}

\begin{proposition}
    Let $C$ be the cantor set. Then \begin{enumerate}
        \item $C$ is compact, nowhere dense, and totally disconnected (i.e. the only connected subsets are single points). Moreover, $C$ has no isolated points.
        \item $m(C) = 0$.
        \item $\on{card}(C) = \mathfrak{c}$, the cardinality of the continuum.
    \end{enumerate}
\end{proposition}

\begin{proof}
    The first claim follows from the base-3 representation, and will appear in the homework. We make a comment, though: disconnected and no isolated points means that for any point $x \in C$ and any $\eps > 0$, we want to show that $B_\eps(x) \not\subset C$ but $B_\eps(x) \cap C \supsetneq \{x\}$.

    By definition, \[m(C) = 1 - \frac{1}{3} - 2\left(\frac{1}{3}\right)^3 - \cdots = 1 - \sum_0^\infty \frac{2^j}{3^{j+1}} = 1 - \frac{1}{3} \left(\frac{1}{1 - 2/3}\right) = 0.\] Next, consider the Cantor function $f : C \to [0,1]$. Note that this function is onto, so $\on{card}(C) = \mathfrak{c}$.
\end{proof}

\begin{remark}
    We make some relevant and not-so-relevant remarks.\begin{enumerate}
        \item Consider $[0,1)$ and scale it by a factor of $1/n$. We need $n$ copies to recover the set, and this is really because the set has dimension 1. Now if you think about $[0,1)^2$ and scale it by a factor of $[0,1)^2$ and scale it by a factor of $1/n$, we need $n^2$ copies to recover the set. Now for the Cantor set, if we scale it by a factor of $1/3$, we need 2 copies. Note that $2 = 3^{\ln 2 / \ln 3}$, and so we can think of this set having dimension $\ln 2 / \ln 3$. We can use this idea to construct a more general definition of dimension.
        \item Going back to the Cantor function, we can extend it to be an increasing function $\ol{f} : [0,1] \to [0,1]$ by defining $f$ as a constant on each interval removed from $C$. Since $\ol{f}$ is increasing and does not have a jump, it is continuous. This function is called the Cantor-Lebesgue function, or the Devil's staircase. Notice that on all intervals in $[0,1] \setminus C$, $\ol{f}$ is contant and hence differentiable almost everywhere. \dhcomment{Review.}
        \item There exist generalizations of the Cantor set, as seen in \citet[Exercise~32]{folland1999real}.
        \item Recall that not every Lebesgue measurable set is a Borel set. The Cantor set is a Borel set as a countable intersection of closed sets, but the Cantor set does have non-Borel subsets. One can argue this by using transfinite induction to show that \[\on{card}(\SL) = \on{card}(2^{[0,1]}) > \mathfrak{c}, \ \ \ \ \text{and also that} \ \ \ \ \on{card}(\SB_\R) = \on{card}([0,1]) = \mathfrak{c}.\]
    \end{enumerate}
\end{remark}

\section{Integration}

With measures in hand, we want to build a robust notion of the integral. 

\subsection{Measurable Functions}
% In particular, we want suitable limits of integrable functions to still be integrable. We start by considering a set $E \in \SM_\mu$ and the characteristic function $\ind_E$. Ideally, we would have $\int \ind_E \d \mu = \mu(E)$. Similarly, for a sequence of disjoint sets $E_j \in \SM_\mu$, we would have $\sum_j \ind_{E_j} = \ind_{\bigcup E_j}$ and $\int \sum \ind_{E_j} \d \mu = \sum_j \mu(E_j)$. We also want $\int 5 \ind_E \d \mu = 5 \mu(E)$. The following question motivates this section. Is it possible to approximate arbitrary integrals using these building blocks?
Recall that any mapping $f : X \to Y$ induces a map $f^{-1} : 2^Y \to 2^X$ by taking $f^{-1}(E) = \{x \in X : f(x) \in E\}$. This operation preserves unions, intersections, and complements. Hence, if $\SN$ is a $\sigma$-algebra on $Y$, the collection $\{f^{-1}(E) : E \in \SN\}$ is a $\sigma$-algebra on $X$. We start by using these ideas to introduce the notion of a measurable function.

\begin{definition}
    Let $(X, \SM)$, $(Y, \SN)$ be measurable spaces. We say the map $f : X \to Y$ is $(\SM, \SN)$-measurable (or just measurable when the spaces are understood) if $f^{-1}(E) \in \SM$ for all $E \in \SN$. Note that this condition implies $\{f^{-1}(E) : E \in \SN\} \subset \SM$.
\end{definition}

\begin{example}
    Consider the functions $f : (X, 2^X) \to (Y, \SN)$ and $g : (X, \SM) \to (Y, \{\emptyset, Y\})$. These are trivial examples of measurable functions.
\end{example}

\noindent It will be useful to know that the condition for this definition can be relaxed into a sufficient condition for measurability of a function.

\begin{proposition}
    If $\SN$ is generated by $\CE$, then $f : X \to Y$ is $(\SM, \SN)$-measurable if and only if $f^{-1}(E) \in \SM$ for all $E \in \CE$.
\end{proposition}

\begin{proof}
    One direction is trivial. To see the other direction, consider the collection $\{E \subset Y : f^{-1}(E) \in \SM\}$. By the properties of the inverse operator, this is a $\sigma$-algebra \dhcomment{check as exercise}. This collection also contains $\CE$, so it must contain $\SN$.
\end{proof}

\begin{corollary}
    If $X$ and $Y$ are topological spaces, then every continuous function $f : X \to Y$ is $(\SB_X, \SB_Y)$-measurable.
\end{corollary}

\begin{proof}
    Let $\CE \subset \SB_Y$ be all open sets in $Y$. By definition of $\SB_Y$, $\CE$ generates $\SB_Y$. By continuity, every set $f^{-1}(E)$ for $E \in \CE$ is open in $X$, so $f^{-1}(E) \in \SB_X$. Invoking the proposition above, the proof is complete.
\end{proof}

\subsubsection{Lebesgue-Measurable Functions}

We now retrict to a regime of interest.

\begin{definition}
    For functions $f : (X, \SM) \to \R$ (resp. $f : (X, \SM) \to \C$), we say a function is $\SM$-measurable if it is $(\SM, \SB_\R)$-measurable (resp. $(\SM, \SB_\C)$-measurable). In particular, we say a function $f : \R \to \C$ is Lebesgue measurable if $f$ is $(\SL, \SB_\C)$-measurable. We say $f : \R \to \C$ is Borel measurable if $f$ is $(\SB_\R, \SB_\C)$-measurable.
\end{definition}

% Note that if we have \[(X, \SM) \overset{f}{\to} (Y, \SN) \overset{g}{\to} (Z, \SO)\] for measurable $f$ and $g$, then the composition $g \circ f$ is $(\SM, \SO)$-measurable. So if $f : \R \to \R$ is Lebesgue measurable and $g : \R \to \R$ is Borel measurable, then $g \circ f$ is Lebesgue measurable. However, this relies crucially on copaitibility of the domains. If $f$ is $(\SB_R, \SB_R)$-measurable and $g$ is $(\SL_R, \SB_\R)$-measurable, then in general, $g \circ f$ is neither Boreal nor Lebesgue measurable. This will appear on the homework.

\begin{remark}
    Why do we not ask for functions to satisfy $(\SL, \SL)$-measurability? This is quite a strong property. A simple example to consider is the Cantor set $C$ and Cantor function $f$. We have seen that $f$ is a bijection into the Cantor set. One can show that $f$ is Lebesgue measurable. However, for any $A \subset [0,1]$, one can cook up a non-measurable subset $F \subset A$ such that $f(F) \subset C$ is a subset of a null set, hence a null set, but $f^{-1}(f(F)) = F$ is non-measurable. So the inverse image of a Lebesgue measurable set by a measurable function need not remain Lebesgue measurable.
\end{remark}

We want to obtain some nice characterizations of function measurability. Since we have shown that the Borel $\sigma$-algebra is generated by the various collections $\CE \subset \SB_\R$, we obtain the following simple yet powerful proposition.

\begin{proposition}\label{prop:measurable-function-elementary-test}
    If $(X, \SM)$ is a measurable space, and $f : X \to \R$, TFAE: \begin{enumerate}
        \item $f$ is $\SM$-measurable;
        \item $f^{-1}((a, \infty)) \in \SM$ for all $a \in \R$;
        \item $f^{-1}([a, \infty)) \in \SM$ for all $a \in \R$;
        \item $f^{-1}((-\infty, a)) \in \SM$ for all $a \in \R$;
        \item $f^{-1}((-\infty, a]) \in \SM$ for all $a \in \R$.
    \end{enumerate}
\end{proposition}

\begin{proof}
    Rays generate $\SB_\R$.
\end{proof}

\begin{definition}
    If $f$ is a function on $(X, \SM)$ and $E \in \SM$, we say $f$ is measurable on $E$ if $f^{-1}(B) \cap E \in \SM$ for all Borel sets $B$. This is equivalent to the requirement that $\restr{f}{E}$ be $\SM_E$-measurable for $\SM_E = \{F \cap E : F \in \SM\}$.
\end{definition}

What if we want to consider functions $f : X \to \R^n$? Let $X$ be a set and $\{(Y_\alpha, \SG_\alpha)\}_{\alpha \in A}$ be a family of measurable spaces and $f_\alpha : X \to Y_\alpha$ for each $\alpha \in A$. Then there exists a unique (smallest) $\sigma$-algebra on $X$ such that all $f_\alpha$ are measurable, ie. the $\sigma$-algebra generated by the collection \[\{f_\alpha^{-1}(E_\alpha) : E_\alpha \in \SG_\alpha, \forall \alpha \in A\}.\] We call this the $\sigma$-algebra generated by the family $\{f_\alpha\}_{\alpha \in A}$. For instance, if $X = \prod_{\alpha \in A} Y_\alpha$ is the product set and $f_\alpha = \pi_\alpha : X \to Y_\alpha$ is the coordinate map, then the $\sigma$-algebra generated by $\{f_\alpha\}_{\alpha \in A}$ is the product $\sigma$-algebra.

\begin{proposition}
    Let $(X, \SM)$ and $(Y_\alpha, \SG_\alpha)$ for $\alpha \in A$ be measurable spaces. Let $Y = \prod_{\alpha \in A} Y_\alpha$, let $\SG = \bigotimes_{\alpha \in A} \SG_\alpha$, and let $\pi_\alpha : Y \to Y_\alpha$ be coordinate maps. Then $f : X \to Y$ is $(\SM, \SG)$-measurable if and only if $f_\alpha = \pi_\alpha \circ f : X \to Y_\alpha$ is $(\SM, \SG_\alpha)$-measurable for all $\alpha \in A$.
\end{proposition}

\begin{proof}
    The coordinate maps $\pi_\alpha$ are measurable. Hence, if $f : X \to Y$ is measurable, then so is the composition $f_\alpha$. Conversely, if each $f_\alpha$ is measurable, then for $E_\alpha \in \SG_\alpha$, we have \[f_\alpha^{-1}(E_\alpha) = (\pi_\alpha \circ f)^{-1}(E_\alpha) = f^{-1}(\pi_\alpha^{-1}(E_\alpha)) \in \SM.\] But $\pi_\alpha^{-1}(E_\alpha)$ are exactly the generators of $\SG$, so $f$ is measurable.
\end{proof}

\begin{corollary}
    A function $f : X \to \C$ is $\SM$-measurable if and only if $\Re f$ and $\Im f$ are $\SM$-measurable. Similarly, $f : X \to \R^n$ is $\SM$-measurable if and only if $f_i$ are all $\SM$-measurable.
\end{corollary}

For functions with singularities, it is useful to allow these functions to take values in $\ol{\R}$. Another reason we may want this is so that for any nonempty set $A \subset \ol{\R}$, the supremum and infemum are well-defined elements of $\ol{\R}$. For our purpose, we just require that $f^{-1}(\{-\infty\})$, $f^{-1}(\{\infty\})$, and the preimages of all usual Borel sets are measurable. Define \[\SB_\R = \{E \subset \ol{\R} : E \cap \R \in \SB_\R\}\] and define $f : X \to \ol{\R}$ to be $\SM$-measurable if it is $(\SM, \SB_{\ol{\R}})$-measurable. \dhcomment{Q: Is this the same as the borel algebra generated by the standard topology on $\ol{\R}$?}

\begin{proposition}\label{prop:add-product-measurable-functions}
    Let $f, g : X \to \C$ be $\SM$-measurable. Then $f + g$ and $f\cdot g$ are also $\SM$-measurable.
\end{proposition}

\begin{remark}
    This proposition is true for $\ol{\R}$-valued functions as well
\end{remark}

\begin{proof}[Proof of \cref{prop:add-product-measurable-functions}]
    We can view $f + g$ as a composition. Define $F : X \to \C \times \C$, $x \mapsto (f(x), g(x))$. Also define $\phi : \C \times \C \to \C$ sending $(z_1, z_2) \mapsto z_1 + z_2$ and $\psi : \C \times \C \to \C$ by $(z_1, z_2) \mapsto z_1 z_2$. Note that both $\phi$ and $\psi$ are $(\SB_{\C \times \C}, \SB_{\C})$-measurable since they are continuous. Next, by our previous proposition, $F$ is $(\SM, \SB_{\C \times \C})$-measurable. Therefore, $f + g = \phi \circ F$ and $f \cdot g = \psi \circ F$ are both $\SM$-measurable.
\end{proof}

Thinking back on the motivations of measure theory, we want to consider measurability of functions under some limiting process. 

\begin{proposition}\label{prop:limit-measurable-functions}
    Let $\{f_j\}$ be a sequence of $\ \ol{\R}$-valued measurable functions on $(X, \SM)$. Then the functions \[g_1 = \sup_j f_j, \ \ \ g_2 = \inf_j f_j, \ \ \ g_3 = \limsup_j f_j, \ \ \ g_4 = \liminf_j f_j\] are measurable. Moreover, if the pointwise limit exists for every $x \in X$, then $\lim_j f_j$ is also measurable.
\end{proposition}

\begin{remark}
    Riemann-integrability is not in general preserved under such limiting processes. \dhcomment{Exercise.}
\end{remark}

\begin{proof}[Proof of \cref{prop:limit-measurable-functions}]
    Consider $x \in g_1^{-1}((a, \infty])$. This is true if and only if $\sup_j f_j(x) > a$, which is true if and only if there exists some $j_0$ such that $f_{j_0}(x) > \alpha$. In particular, we have \[g_1^{-1}((a, \infty]) = \bigcup_{j=1}^{\infty} f_j^{-1}((a, \infty]).\] This is a coutable union of measurable sets, so measurable. That is, $g_1$ is measurable by \cref{prop:measurable-function-elementary-test}. Similarly, \[g_2^{-1}([-\infty, a)) = \bigcup_{j=1}^{\infty} f_j^{-1}([-\infty, a)).\]

    For $g_3$, note that limsup can be equivalently using only the supremum and infemum operators. For $h_k(x) = \sup_{j > k} f_j(x)$, $h_k$ is measurable for each $k$. Hence, \begin{align*}
        g_3(x) = \limsup_{j} f_j(x) &= \inf_k \sup_{j > k} f_j(x) = \inf_k h_k(x),
    \end{align*} which is measurable. A similar argument works for $g_4$.

    Finally, $\lim_{j \to \infty} f_j(x)$ exists and equals the $\liminf$ (or $\limsup$) if and only if they exist and $\limsup_j f_j(x) = \liminf_j f_j(x)$. Hence, if the limit exists for all $x$, the limit is equivalent to the liminf, which exists and is measurable.
\end{proof}

\begin{corollary}
    If $f, g : X \to \ol{\R}$ are measurable. Then so are $\max(f, g)$ and $\min(f, g)$.
\end{corollary}

\begin{corollary}
    If $\{f_j\}$ is a sequence of complex-valued measurable functions and $f(x) = \lim_{j \to \infty} f_j(x)$ exists for all $x$, then $f$ is measurable.
\end{corollary}

\subsection{Simple Functions}

\begin{definition}
    For $f : X \to \ol{\R}$, define $f^+(x) = \max(f(x), 0)$ and $f^-(x) = \max(-f(x), 0)$ as the positive and negative parts of $f$, respectively. 
\end{definition}

\begin{remark}
    A note on this definition.
    \begin{itemize}
        \item Note that $f^+, f^- \geq 0$ and that $f = f^+ - f^-$.
        \item If $f$ is measurable, then so are $f^+$ and $f^-$. Moreover, if $f^+$ and $f^-$ are both measurable, then $f$ is measurable.
        \item For $f : X \to \C$, we can come up with an analogous decomposition, the polar decomposition: \[f = (\sgn f) |f|, \ \ \text{where} \ \ \sgn z = \begin{cases}
            \frac{z}{|z|} &\text{if} \ |z| \neq 0\\
            0 &\text{else}.
        \end{cases}\] It holds that $f$ is measurable if and only if $\sgn f$ and $|f|$ are measurable. Indeed, if $f$ is measurable, since $|\cdot|$ is continuous and $f$ is measurable, $|f|$ is measurable. Next, $\sgn$ is continuous except at the origin. If $U \subset \C$ is open, then $\sgn^{-1}(U)$ is either open or of the form $V \cup \{0\}$ where $V$ is open. So $\sgn$ is Borel-measurable and therefore $\sgn f$ is measurable. The other direction is clear by previous propositions.
        \item For a set $E \subset X$, then the characteristic function $\chi_E$ is measurable if and only if $E \in \SM$.
    \end{itemize}
\end{remark}

\begin{definition}
    A simple function on $X$ is a finite linear combination (in general, with complex coefficients) of characteristic functions of measurable sets $E \in \SM$. We do not allow the coefficients to take values $\pm \infty$.
\end{definition}

\begin{remark}
    One can characterize simple functions in the following way. Suppose $f : X \to \C$ is simple; it takes only finitely many values $\on{range}(f) = \{z_1, \ldots, z_n\}$. Also assume that $f$ is measurable. Then \[E_j := f^{-1}(\{z_j\}) \in \SM.\] Moreover, we can write \[f = \sum_{j=1}^n z_j \chi_{E_j}, \ \ \ \ \text{for all $E_j$ disjoint}.\] We call this the standard representation of $f$.
\end{remark}

It turns out that we can approximate measurable functions using simple functions, and this will be a very useful result for integration.

\begin{theorem}\label{thm:simple-function-approximation}
    Let $(X, \SM)$ be a measurable space. \begin{enumerate}
        \item If $f : X \to [0,\infty]$ is a nonnegative measurable function, then there exists a sequence $\{\phi_n\}_n$ of simple functions such that $0 \leq \phi_1 \leq \phi_n \leq \cdots \leq f$ such that $\phi_n \to f$ pointwise, $\phi_n \rightrightarrows f$ (uniformly) on any set on which $f$ is bounded.
        \item Similarly, if $f : X \to \C$ is measurable, then there exists a sequence of simple functions $\{\psi_n\}_n$ such that $0 \leq |\psi_1| \leq |\psi_2| \leq \cdots \leq |f|$ and $\psi_n \to f$ pointwise and $\psi_n \rightrightarrows f$ on any set where $f$ is bounded.
    \end{enumerate}
\end{theorem}

\begin{proof}
    We use an explicit construction. For $n \in \N$, and $0 \leq k \leq 2^{2n} - 1$. Let $E_n^k = f^{-1}((k 2^{-n}, (k+1)2^{-n}])$ and $F_n = f^{-1}((2^n, \infty])$. Now define \[\phi_n = \left(\sum_{k=0}^{2^{2n}-1} k 2^{-n} \cdot \chi_{E_n^k} \right) + 2^n \chi_{F_n}.\] This is a simple function. 
    
    It remains to check the convergence claims. First notice that for all $x \in X$, $\phi_n(x)$ is monotonically increasing in $n$. Moreover, for each $n$, we have $0 \leq f - \phi_n \leq 2^{-n}$ on the set where $f \leq 2^{n}$. Hence, for any fixed $x$, we have pointwise convergence, and for any set where $f$ is bounded, we have uniform convergence., 

    The second claim follows by writing $f = g + ih$ and applying the first result of $g^+$, $g^-$, $h^+$ and $h^-$ to get approximations $\psi_{g, n}^+$, $\psi_{g, n}^-$, $\psi_{h, n}^+$, $\psi_{h, n}^-$. We can then let $\psi_n = (\psi_{g, n}^+ - \psi_{g, n}^-) + i(\psi_{h, n}^+ - \psi_{h, n}^-)$.
\end{proof}

\begin{remark}
    Consider the series
    \[\sum_{k=0}^{2^{2n} - 1} k2^{-n} \mu\Big( f^{-1}\Big((k2^{-n}, (k+1)2^{-n})\Big)\Big).\] This corresponds to a Riemann sum for the integral \[\int_0^{2^n} \mu(\{x : f(x) > t\}) \d t.\] In particular, \[\int f \d \mu = \int_0^\infty \mu(\{x : f(x) > t\}) \d t,\] as we will see.
\end{remark} 

\begin{proposition}
    The following implications are valid if and only if $\mu$ is a complete measure. \begin{enumerate}
        \item If $f$ is measurable and $f = g$, $\mu$-a.e., then $g$ is measurable.
        \item If $f_n$ is measurable for $n \in \N$ and $f_n \to f$, $\mu$-a.e., then $f$ is measurable.
    \end{enumerate}
\end{proposition}

\begin{proof}
    Exercise.
\end{proof}

\begin{proposition}\label{prop:equal-ae-function-complete-measure}
    Let $(X, \SM, \mu)$ be a measure space and $(X, \ol{\SM}, \ol{\mu})$ be its completion. If $f$ is $\ol{\SM}$-measurable function on $X$ into $[0,\infty]$ or $ \C$, then there exists an $\SM$-measurable function $g$ such that $f = g$, $\ol{\mu}$-a.e.
\end{proposition}

\begin{proof}
    Recall that $\ol{\SM}$ differs from $\SM$ only by null sets. If $f = \chi_E$ where $E \in \ol{\SM}$, then $E = F \cup N$ where $F \in \SM$ and $\ol{\mu}(N) = 0$. So, take $g = \chi_F$, and note that (1) $f = g$, $\ol{\mu}$-a.e. and that (2) $g$ is $\SM$-measurable. We can say something similar about simple functions. 

    In the general case, we aim to approximate with simple functions. Choose a sequence $\{\phi_n\}_n$ of $\ol{\SM}$-measurable simple functions which converge to $f$. For each $n$, let $\psi_n$ be an $\SM$-measurable simple function such that $\phi_n = \psi_n$, $\ol{\mu}$-a.e.; these only differ on a set $N_n \in \ol{\SM}$ where $\ol{\mu}(N_n) = 0$. Then note that $\bigcup_1^\infty N_n$ is still a null set in $\ol{\SM}$. Since $\ol{\mu}$ is a completion of $\mu$, there is some set $N \in \SM$ such that $\bigcup_1^\infty N_n \subset N$ and $\mu(N) = 0$. Set \[g = \lim_{n \to \infty} \chi_{(X \setminus N)} \psi_n,\] which is $\SM$-measurable by \cref{prop:limit-measurable-functions}, and $g = f$ on $N^c$.
\end{proof}

\subsection{Integration of Nonnegative Functions}

Fix a measure space $(X, \SM, \mu)$. 

\begin{definition}
    We define $L^+$ to be the space of all measurable functions $X \to [0,\infty]$. If $\phi$ is a simple function with standard representation $\phi = \sum_{j=1}^n a_j \chi_{E_j}$, then we define the integral of $\phi$ with respect to $\mu$ by the expression \[\int \phi \d \mu = \sum_1^n a_j \mu(E_j).\] As a convention, $0 \cdot \infty = 0$ within the sum.
\end{definition}

\begin{remark}
    Some notes on this definition.
    \begin{itemize}
        \item Note that $\int \phi \d \mu = \infty$ is allowed because $\mu(E_j)$ need not be finite. 
        \item Since $\phi \in L^+$, there is no ambiguity for indefinite sign $\infty - \infty$.
        \item Different notations include \[\int\phi = \int \phi \d \mu = \int \phi(x) \d \mu(x) = \int \phi(x) \mu(\d x) = E_\mu[\phi].\]
        \item There is an immediate definition for integrals over domains. If $A \in \SM$, then $\phi \cdot \chi_A$ is also simple. We define the integral over $A$ as \[\int_A \phi \d \mu = \int \phi \cdot \chi_A \d \mu.\]
        \item Such integrals also have several notations: \[\int_A \phi \d \mu = \int_A \phi = \int_A \phi(x) \d \mu(x), \ \ \ \text{and} \ \ \ \int \phi = \int_X \phi.\]
    \end{itemize}
\end{remark}

We now discuss several properties of these integrals.

\begin{proposition}
    Let $\phi, \psi \in L^+$ be simple functions. \begin{enumerate}
        \item If $c \geq 0$, then $\int c \phi = c \int \phi$.
        \item $\int (\phi + \psi) = \int \phi + \int \psi$.
        \item If $\phi \leq \psi$, then $\int \phi \leq \int \psi$.
        \item For fixed $\phi \in L^+$, the map $A \mapsto \int_A \phi \d \mu$ is a measure on $\SM$.
    \end{enumerate}
\end{proposition}

\begin{proof}
    The proof of (a) is straightforward. To see (b), let $\phi = \sum_1^n a_j \chi_{E_j}$ and $\psi = \sum_1^m b_k \chi_{F_k}$. We write $E_{jk} = E_j \cap F_k$, and note that we have disjoint unions $\bigsqcup_k E_{jk} = E_j$ and $\bigsqcup_j E_{jk} = F_k$. Then \[\phi = \sum_{j, k} a_j \chi_{E_{jk}}, \ \ \ \psi = \sum_{j, k} b_k \chi_{E_{jk}}.\] Hence, \begin{align*}
        \int \phi + \int \psi &= \sum_j a_j \mu(E_j) + \sum_k b_k \mu(F_k) \\
        &= \sum_{j,k} (a_j + b_k) \mu(E_j \cap E_k) = \int (\phi + \psi).
    \end{align*}
    To see (c), note that $\phi \leq \psi$ implies $a_j \leq b_k$ wherever $E_j \cap E_k \neq \emptyset$. Hence, \begin{align*}
        \int \phi &= \sum_{j, k} a_j \mu(E_{jk}) \\
        &\leq \sum_{j,k} b_k \mu(E_{jk}) = \int \psi.
    \end{align*}
    Finally, for (d), we must check: \begin{itemize}
        \item $\int_\emptyset \phi = 0$;
        \item $\sigma$-additivity.
    \end{itemize}
    The first point follows from the observation that \[\int_\emptyset \phi = \sum_j a_j \mu(E_j \cap \emptyset).\] To see the second point, let $\{A_j\}$ be a disjoint sequence in $\SM$ and $A = \bigcup_j A_j$. Now, \begin{align*}
        \int_A \phi = \sum_j a_j \mu(A \cap E_j) &= \sum_j a_j \mu\left(\bigcup_k (A_k \cap E_j)\right)\\
        &= \sum_j a_j \sum_k \mu(A_k \cap E_j)\\
        &= \sum_k \sum_j a_j \mu(A_k \cap E_j) = \sum_k \int_{A_k} \phi.
    \end{align*}
\end{proof}

\begin{definition}
    Let $f \in L^+$ but not necessarily simple. We define the integral of $f$ via the expression \[\int f \d \mu = \sup \left\{ \int \phi \d \mu : 0 \leq \phi \leq f, \phi \ \text{simple} \right\}.\]
\end{definition}

\begin{remark}
    We make some comments on this definition before proceeding. \begin{itemize}
        \item If $f$ is simple, this agrees with the definition of the integral of a simple function.
        \item If $f \leq g$, then $\int f \leq \int g$.
        \item For $c \geq 0$, $\int c f = c \int f$.
    \end{itemize}
\end{remark}

\begin{theorem}[Monotone Convergence Theorem]
    If $\{f_n\} \subset L^+$ satisfies $f_j \leq f_{j+1}$ for all $j$, and $f = \lim_{n \to \infty} f_n (= \sup_n f_n)$, then \[\int f = \lim_{n \to \infty} \int f_n.\]
\end{theorem}

\begin{proof}
    For any $x$, $\{f_n(x)\}$ is an increasing sequence. Hence, $\lim_n f_n(x) = f(x)$ equals the supremum, so $f$ is measurable. Since $f_n \leq f$ pointwise, we have \[\lim_{n \to \infty} \int f_n \leq \int f.\]

    It remains to show the reverse inequality. Using the definition of $f$, let $\phi$ be any simple function $0 \leq \phi \leq f$. Now for any $\alpha \in (0,1)$, consider \[E_n = \{x : f_n(x) \geq \alpha \phi(x)\} .\] Since $f_n$ is increasing, $\{E_n\}$ is an increasing sequence of measurable sets whose union is $X$. We also have that for any $n$, \[\int f_n \geq \int_{E_n} f_n \geq \alpha \int_{E_n} \phi.\] On the other hand, since $\int_{(\cdot)}\phi : \SM \to [0,\infty]$ is a measure, continuity of measures implies that, \[\lim_{n \to \infty} \alpha \int_{E_n} \phi = \alpha \int \phi.\] That is, \[\lim_{n \to \infty} \int f_n \geq \alpha \int \phi.\] Note that $\alpha$ and $\phi$ are arbitrary, so by taking $\alpha \to 1$ and the supremum over all simple functions such that $0 \leq \phi \leq f$, we have that \[\lim_{n \to \infty} \int f_n \geq \int f.\]
\end{proof}

\begin{remark}
    The definition of $\int f \d \mu$ was via all possible sequences $\phi$, a huge family. Now, MCT tells us that it is sufficient to compute $\lim \int \phi_n$ where $\phi_n$ where $\phi_n$ are simple and $\phi_n \nearrow f$. This often allows us to establish proofs by looking only at simple functions.
\end{remark}

\begin{theorem}
    If $\{f_n\} \subset L^+$ is a finite or contably infinite sequence, and $f = \sum_n f_n$, then \[\int f = \sum_n \int f_n.\]
\end{theorem}

\begin{proof}
    Let $f_1, f_2 \in L^+$. Then we can find $\phi_j^1$ and $\phi_j^2$ approximating $f_1$ and $f_2$, respectively. Next, \[\int f_1 + f_2 \underset{\text{MCT}}{=} \lim_j \int \phi_j^1 + \phi_j^2 = \lim_j \left( \int \phi_j^1 + \int \phi_j^2 \right) = \lim_j \int \phi_j^1 + \lim_j \int \phi_j^2 \underset{\text{MCT}}{=} \int f_1 + \int f_2.\] By induction, for any finite sum, $\int \sum f_n = \sum \int f_n$. Now, consider the partial sum $g_N = \sum_1^N f_n$ and apply MCT to this function. We obtain \[\int \sum_1^\infty f_n = \sum_1^\infty \int f_n.\]
\end{proof}

\begin{proposition}
    If $f \in L^+$, then $\int f = 0$ if and only if $f = 0$ a.e.
\end{proposition}

\begin{proof}
    First assume that $f$ is simple. Then $f = \sum_1^n a_j \chi_{E_j}$, $a_j \geq 0$. Observe that $\int f = 0$ holds if and only if for each $j$, $a_j = 0$ or $\mu(E_j) = 0$. Now let $f \in L^+$ be given. Let $\phi$ be a simple function such that $0 \leq \phi \leq f$. Assuming $f = 0$ a.e., we have that $\phi = 0$ a.e. Hence, \[\int f = \sup_{0 \leq \phi \leq f} \int \phi = 0.\]

    Going the other direction, if $\{x : f(x) > 0\}$ is not a null set, then we can write \[\{x : f(x) > 0\} = \bigcup_1^\infty E_n \ \ \ \ \text{where} \ \ \ \ E_n = \{x : f(x) > 1/n\}.\] One such $E_n$ must not be a null set. But then $f$ is bounded below by a simple function, $f > \frac{1}{n} \chi_{E_n}$. In particular, $\int f > 0$, which contradicts the assumption that $\int f = 0$.
\end{proof}

\begin{corollary}
    If $\{f_n\} \subset L^+$, $f \in L^+$, and $f_n(x) \nearrow f(x)$ a.e. in $X$, then \[\int f = \lim_{n \to \infty} \int f_n.\]
\end{corollary}

\begin{proof}
    Suppose $f_n(x)$ increases to $f(x)$ for $x \in E$ and $\mu(E^c) = 0$. We define $g_n = f_n \chi_E \nearrow g = f \chi_E$ for every $x$. Since $f_n = g_n$ a.e. and $g = f$ a.e., the previous proposition gives \[\int f = \int g \underset{\text{MCT}}{=} \lim_n \int g_n = \lim_n \int f_n.\]
\end{proof}

\subsection{Fatou's Lemma and Standard Counterexamples}

What happens when we drop the monotonicity assumption?

\begin{example}[Escape to infinity]
    Define $f_j = \chi_{[j, j+1)}$. It is clear $f_j \in L^+$ and $\int f_j \d m = 1$. Note that $f_j(x) \to 0$ for every $x$. Hence, $\int \lim_j f_j = 0$, but $\lim_j \int f_j = 1$.
\end{example}

\begin{example}[Concentration/blowup]
    Define $f_j(x) = j \chi_{[0, 1/j)}$. So $f_j \in L^+$ with $\int f_j \d m = 1$. Note that $f_j(x) \to 0$ for all $x > 0$, but $f_j(0) \to \infty$. Again, $0 \neq 1$. 
\end{example}

\noindent In both of these examples, we have the inequality \[\underbrace{0}_{\int \lim} < \underbrace{1}_{\lim \int}.\] One can make a more general statement, which gives a partial characterization of our counterexamples.

\begin{lemma}[Fatou]
    If $\{f_n\} \subset L^+$, then \[\int \liminf f_n \leq \liminf \int f_n.\]
\end{lemma}

\begin{proof}
    Recall that $\liminf_n = \sup_k \inf_{n \geq k}$. For each $k \geq 1$, we have \[\inf_{n \geq k} f_n \leq f_j, \ \ \ \text{for all} \ \ j \geq k.\] Now, \[\int \inf_{n \geq k} f_n \leq \int f_j \ \ \ \text{for all} \ \ j \geq k, \ \ \ \text{and} \ \ \ \int \inf_{n \geq k} f_n \leq \inf_{j \geq k} \int f_j.\] But note that $\inf_{n \geq k} f_n \nearrow \liminf f_n$ as $k \to \infty$, so by MCT, we have \[\int \liminf f_n = \lim_{k \to \infty} \int \inf_{n \geq k} f_n \leq \liminf \int f_n.\]
\end{proof}

\begin{corollary}
    Let $\{f_n\} \subset L^+$, $f \in L^+$, and $f_n \to f$ a.e. Then \[\int f \leq \liminf \int f_n.\]
\end{corollary}

\begin{proof}
    We have $f = \lim_{n \to \infty} f_n$ a.e., so $f = \liminf f_n$ a.e. Using the proposition about integrals of functions that are equivalent except for a null set, we have \[\int f = \int \liminf f_n \leq \liminf \int f_n.\]
\end{proof}

\subsection{Integration of Complex-Valued Functions}

\begin{definition}
    If $f : X \to \R$ is measurable and at least one of $\int f^+$ or $\int f^-$ is finite, we define \[\int f = \int f^+ - \int f^-.\] If both $\int f^+$ and $\int f^-$ are finite, we say $f$ is integrable.\footnote{One must not confused the two notions introduced here: the integral $\int f$ may exist even when $f$ is not integrable.}
\end{definition}

\begin{proposition}
    A function $f : X \to \R$ is integrable if and only if $\int |f| < \infty$.
\end{proposition}

\begin{proposition}
    The set of all integrable, real-valued functions on $X$ is a real vector space, and the integral is a linear functional on this vector space.
\end{proposition}

\begin{proof}
    To start, note that $|af + bg| \leq |a||f| + |b||g|$, so linear combinations will be integrable. To show linearity of the mapping $\int (\cdot) : V \to \R$, we check the necessarily conditions. We have $\int a f = a \int f$ using the definitions. Now let $h = f + g$ for integrable $f$, $g$. We have \[h = h^+ - h^- = f^+ + g^+ - f^- - g^-.\] We also have \[h^+ + f^- + g^- = h^- + f^+ + g^+ \in L^+.\] By the theorem about summing functions in $L^+$, we have \[\int h^+ + \int f^- + \int g^- = \int h^- + \int f^+ + \int g^+,\] and \[\int h = \int h^+ - \int h^- = \int f^+ - \int f^-  +\int g^+ - \int g^- = \int f + \int g.\]
\end{proof}

\begin{definition}
    If $f : X \to \C$ is measurable, we say that $f$ is integrable if $\int |f| < \infty$. For $E \in \SM$, we say that $f$ is integrable in $E$ if $\int_E |f| < \infty$. We define \[\int f = \int \Re f + i \int \Im f.\]
\end{definition}

\begin{remark}
    For $f : X \to \C$, one can verify that \[\Re \int f = \int \Re f, \ \ \ \Im \int f = \int \Im f.\]
\end{remark}

\begin{proposition}
    The function $f: X \to \C$ is integrable if and only if $\Re f$ and $\Im f$ are integrable. 
\end{proposition}

\begin{proof}
    This follows from the fact that \[|f| \leq |\Re f| + |\Im f| \leq 2|f|.\]
\end{proof}

\begin{proposition}
    The space of complex-valued integrable functions is a complex vector space, and the integral is a complex linear funcitonal on this vector space.
\end{proposition}

\begin{proof}
    Using previous proposition.
\end{proof}

We denote the vector space of complex-valued integrable functions under measure $\mu$ by $L^1(\mu)$ (or $L^1(X, \mu)$ or $L^1(X)$ or $L^1$). We will eventually redefine $L^1(\mu)$ by quotienting out equivalencies almost everywhere. For now, though, take $L_1(\mu)$ as described.

\begin{proposition}
    If $f \in L^1$, then $\left| \int f \right| \leq \int |f|$.
\end{proposition}

\begin{proof}
    For real-valued functions, \[\left|\int f\right| = \left| \int f^+ - \int f^-\right| \leq \int f^+ + \int f^- = \int f^+ + f^- = \int |f|.\] If $f$ is complex-valued and $\int f = 0$, the claim is trivial. If $\int f \neq 0$, let $\alpha = \ol{\sgn \left(\int f\right)}$. Then \[\int f = \sgn \left(\int f \right) \left|\int f\right|.\] Then \[\alpha \int f = \ol{\sgn \left(\int f\right)} \sgn \left(\int f \right) \left|\int f\right| \implies \int \alpha f = \left|\int f\right|.\] Since $\int \alpha f$ is real then, \begin{align*}
        \left|\int f\right| = \Re \int \alpha f = \int \Re(\alpha f) \leq \int |\alpha f| = \int |f|.
    \end{align*}
\end{proof}

\begin{proposition}
    \begin{enumerate}
        \item If $f \in L^1$, then $\{x : f(x) \neq 0\}$ is $\sigma$-finite.
        \item If $f, g \in L^1$, then $\int_E f = \int_E g$ for all $E \in \SM$ if and only if $\int |f - g| = 0$ if and only if $f = g$ a.e.
    \end{enumerate}
\end{proposition}

\begin{proof}
    Part (i) is left as an exercise. For (ii), suppose that $\int |f - g| = 0$. Then \begin{align*}
        \left| \int_E f - \int_E g \right| = \left|\int_E f-g\right| \leq \int |f-g| \chi_E \leq \int |f-g| = 0.
    \end{align*} That is, $\int_E f = \int_E g$ for all $E \in \SM$. On the other hand, suppose $\int |f - g| > 0$. Then let \[u = \Re(f-g), \ \ \ v = \Im (f-g).\] Then at least one of $u^+$, $u^-$, $v^+$, $v^-$ has to be nonzero on a set of positive measure. W.l.o.g. let $E = \{x : u^+ > 0\}$, which has nonzero measure. Then \[\Re\left(\int_E f - \int_E g\right) = \int_E u^+ > 0,\] where the first equality holds because $u^- = 0$ on $E$. The other equivalence follows from our previous proposition about equality of integrals of functions in $L^+$. 
\end{proof}

We may now define $L^1$ properly.

\begin{definition}
    Define $L^1(\mu)$ to be the set of equivalence classes of a.e.-defined integrable (complex-valued) functions defined on $X$, where $f \sim g$ if and only if $f = g$ $\mu$-a.e.
\end{definition}

\begin{remark}
    $L^1(\mu)$ is still a complex vector space. Although we will henceforth view $L^1(\mu)$ as a space of equivalence classes, we shall still employ the notation $f \in L^1(\mu)$ to mean that $f$ is an a.e.-defined integrable function. We make some further observations.\begin{itemize}
        \item There is a bijective corresondence between $L^1(\mu)$ and $L^1(\ol{\mu})$ by \cref{prop:equal-ae-function-complete-measure}.
        \item $L^1$ is a metric space with metric $\rho(f, g) = \int |f-g|$.
        \item We refer to convergence with respect to the metric $\rho$ as \textit{convergence in $L^1$}; thus $f_n \to f$ in $L^1$ if and only if $\int |f_n - f| \to 0$.
    \end{itemize}
\end{remark}




\pagebreak

\small

\bibliography{refs}
\bibliographystyle{apalike}

\end{document}

  